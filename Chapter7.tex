% Chapter 7

\chapter{Adaptive Modulation Order Selection} % Main chapter title

\label{Chapter7} % For referencing the chapter elsewhere, use \ref{Chapter1} 

\lhead{Chapter 7. \emph{Adaptive Modulation Order Selection}} % This is for the header on each page - perhaps a shortened title
Until now, we have reviewed only two degrees of freedom for assigning transmission rates : time (different OFDM symbols) and frequency (different subcarriers). Now, we introduce modulation order as a third degree of freedom, giving the possibility of encode higher number of bits in order to decrease the latencies of packets, at the expense of higher power consumption. We provide in this chapter 2 dynamic modulation order selection algorithms : Maximum delay bounded scheduler and average delay bounded scheduler, both inspired from the literature for generic wireless communications \cite{khojastepour2003power}\cite{rajan2004delay}. They give a trade-off between latency and power consumption relatively to BPSK case, where we do not impose a real transmission power in Watts. Next in this chapter, an information theoretic analysis investigates required transmission powers to achieve certain capacities on different transmission line topologies. 

\section{Delay-Power Trade-off}

A widely known rule of thumb of digital communications is the need for exponentially larger power transmission for higher rates. If we project this rule on a constellation diagram, we see that the Additive White Gaussian Noise (AWGN) drifts locations of received symbols, which causes decoding errors. Since his seminal paper in 1949 \cite{shannon1949mathematical}, Claude Shannon had founded a new paradigm in telecommunications, \textit{Information Theory}, which sets a bound on the achievable rate for a communications channel with arbitrarily small error probability, given a specific Signal-to-Noise Power Ratio (SNR). Another way to approach this problem is the other way around, i.e. given a desired probability of error, what is the required minimum transmission power. The well known transmission capacity formula from Shannon is as :

\begin{align}
C_{0} =  Blog_{2}(1 + {P_{R} \over P_{N}}) 
\end{align}

where $C_{0}$ is capacity in bits/sec, B is bandwidth in Hz, $P_{R}$ is received signal power in Watts and $P_{N}$ is the ambient white noise power in Watts. $C = {C_{0} \over B} $ is spectral capacity density in bits/sec/Hz. $P_{N}$ can be calculated as : 

\begin{align}
P_{N} =  BN_{0}
\end{align}

where $N_{0}$ is the noise spectral density in Watts/Hz (scalar), which we assume as -174 dBm/Hz at room temperature (290 K). $P_{R}/P_{N}$ is referred as received Signal-to-Noise Ratio (SNR). Note that even though $N_{0}$ is called noise power spectral density, we have chosen to denote the capacity spectral density as $C$, and capacity as $C_{0}$. This is because, we use capacity spectral density as a more important metric, which is associated to modulation orders. 

After manipulating these equations, we can achieve the required minimum received signal power for achieving a transmission rate capacity (Note that we are using information theoretic rate capacity ($C_{0}$) and transmission rate ($R$) as interchangable terms ) :

\begin{align}
{P_{R} \over P_{N}} = {2}^{C \, B}-1
\end{align}

In case of quantized digital communications, such as OFDM, we can transform this equation to the relation, showing the ratio of minimum transmission power required with increasing modulation order (i.e. rate-\textit{bits/subcarrier/symbol}) compared to minimum modulation order BPSK (1 bit/subcarrier/symbol), where $M$ is the modulation order rate (e.g. if 16-QAM : 4 bits/subcarrier/symbol) :

\begin{align}
{P_{R} \over P_{N}} = {2}^{M \, B}-1
\end{align}	


We have seen that required power transmission is an exponentially increasing cost for higher rates. Taking into account how energy is a valuable resource, a voluminous literature on energy efficient optimal rate selection has been created by researchers, from cellular communications to satellite communications, from optical fiber connections to wireless sensor networks \cite{bandyopadhyay2003energy}\cite{fu2003optimal}\cite{zhang2010energy}\cite{berggren2004energy}. \textit{Rate}, actually, is a tool, not the ultimate goal. In other words, a designer would like to control the metrics of Quality-of-Service (QoS) such as average latency, maximum delay, maximum buffer length required, probability of buffer overflow etc., which are results of the selected rate. It is interesting to see that, optimality of delay-power relation is rather an under-exploited subject. The main idea is simple : packets in queues can be delayed in order to send them in longer durations, with slower rates (lower modulation orders), thus to save power. Fig. 7.1 shows the typical Pareto like delay-power relation, which can be potentially exploited.


\begin{figure}[htbp]
  \centering
    \includegraphics[width=0.8\textwidth]{./Chapter6_Figures/delayPowerRelation.eps}
    \rule{35em}{0.5pt}
  \caption[Typical Convex Delay-Power relation]{Typical Convex Delay-Power relation, which gives designer to exploit large energy savings by trading latency} 
  \label{fig:Electron}
\end{figure}



Various researchers approached this issue from different perspectives. At first, parameters and metrics of interest are relatively defined; i.e. one may desire to minimize average delay while setting a bound for the transmission power budget, or vice-versa. One other may wish to set bounds on the maximum delay due to Quality-of-Service requirements or maximum transmission power due to circuit restrictions. In addition, this rate scheduler shall be designed based on different paradigms depending on the mode of transmission, environment, nature of traffic etc. In this chapter, we provide 2 dynamic modulation order selection algorithms for WiNoCoD, trying to minimize the average energy expenditure, while first one is setting a bound on maximum delay of the packets, second one average delay of the packets. 

\section{Adaptive Modulation Option in OFDMA and Delay }

Another revolutionary advantage of using OFDMA in WiNoCoD is the option of using higher modulation orders dynamically on different symbols and subcarriers. In addition to frequency and time, this option provides a third dimension of flexibility. In design phase of WiNoCoD project, 4 modulation orders : BPSK, QPSK, 16-QAM and 64-QAM are selected to be implemented, which provides 1,2,4 and 6 bits per subcarrier respectively. It was shown in M. Hamieh's work that, for our transmission line (see Section 3.3.2.3), between two farthest tilesets, minimum required power for transmission on a single subcarrier with a bit error rate set to $10^{-8}$ using BPSK is -88 dBm and using 64-QAM is -74 dBm \cite{hamieh2014sizing}. Noise factors of receivers are supposed to be 3 dB over whole spectrum. Therefore, using 64-QAM needs 14 dB more power than using BPSK, in other words approximately more than 25 times in linear scale. However, in our simulations while testing our proposed solutions, we will stick to the equation 7.1, which gives the information theoretic rate-power relation. This would allow any kind of different channel coding techniques to be applied, without considering a specific bit or packet error rate. In addition, we have decided to employ modulation orders up to 256-QAM, including also 8-PSK, 32-QAM and 128-QAM, which may be available in further phases of the project. While testing our simulations, we considered the power consumption metric as the linear ratio to the power of lowest modulation order, BPSK. In this thesis work, adaptive modulation order selection is performed on the dynamic bandwidth allocation algorithms, that reconfiguration in all 3 dimensions are exploited.


\section{Decentralized and Centralized Modulation Order Selection Policy}

\subsection{Decentralized Modulation Order Selection}

In decentralized algorithms, just as in previous chapters, on the first symbol of each frame, tilesets broadcast their QSIs. After each tileset receives information on buffer lengths of each other tileset, all tilesets process the same algorithm on the same QSI data to reach the same decision on the arbitration of subcarriers. This process is assumed to take 1 frame length, thus new configuration of the bandwidth is activated for the next frame. We introduce here a set of decentralized dynamic modulation order selection policies on to previously presented bandwidth allocation framework, where at each frame, each tileset decides on its modulation order for the next frame using its QSI and also newly calculated number of subcarriers to be used in next frame. As other tilesets must learn this chosen modulation order to successfully decode this tileset's transmissions on its dedicated RBs, after deciding on the modulation order each tileset encodes its modulation order on the last symbol of the current frame. The possibility for computing and activating different modulation orders on each frame (coherent with the framed structure explained in previous chapters) in a decentralized setting, is illustrated in Fig. 7.2. 

\begin{figure}[htbp]
  \centering
    \includegraphics[width=1.0\textwidth]{./Chapter6_Figures/Chapter6_07.pdf}
    \rule{35em}{0.5pt}
  \caption[Decentralized RB allocation and modulation order selection in WiNoCoD]{Framed structure of decentralized RB allocation and modulation order selection in WiNoCoD} 
  \label{fig:Electron}
\end{figure}

Note that, a tileset needs the total number of RBs it will use during the next frame. OFDMA gives the possibility to encode different modulation orders on each subcarrier, every symbol. However, in an on-chip environment like WiNoCoD with bandwidth and computation restrictions, we constructed a paradigm, where a tileset decides on its modulation order to be used, through whole next frame. Therefore, if a tileset has S RBs (say which can serve S flits with the lowest modulation order, BPSK) in next frame, and if it decides to use a modulation order of 16-QAM, it can serve up to 4S flits. Hence, modulation order selection should be done carefully, provisioning power expenditure, based on the instantaneous QSI and traffic characteristics. Remind that, a tileset selects its modulation order based on the total number of RBs it will have in next frame, therefore through the current frame, it shall perform this operation after the computation of bandwidth allocation. After selecting its modulation order, a tileset broadcasts this modulation order on dedicated subcarriers, so that other tilesets which decode RBs of this tileset can use this moduation order for successful decoding. As you can see, the tileset receivers should activate this operation, \textit{matching RBs and decoding modulation order} via a circuitry or processing unit, just in one symbol. In next symbol, the new frame starts with the new RB arbitration and used modulation orders. We also assume, tilesets are able to tune their transmission powers in one symbol duration.    


Another important point is that each tileset, after deciding its modulation order, should tune their transmission power according to this new modulation order. Fig. 7.3 illustrates the process of power tuning and constellation mappings in a tileset's OFDMA transmission chain. 

\begin{figure}[htbp]
  \centering
    \includegraphics[width=0.8\textwidth, height= 0.55\textwidth]{./Chapter6_Figures/Transmission_Power_2_v2.pdf}
    \rule{35em}{0.5pt}
  \caption[Transmission power tuning for the chosen constellation order]{Transmission power tuning for the chosen constellation order} 
  \label{fig:Electron}
\end{figure}

\subsection{Centralized Modulation Order Selection}

For the centralized modulation order selection, Central Intelligent Unit (CIU) is also responsible for the chosen modulation order of each tileset. After acquiring the QSIs of the tilesets, CIU determines the number of RBs allocated to each tileset for the next frame, as explained in previous chapters. However, this time, after calculating the number of RBs allocated to each tileset, it determines the selected modulation order for the next frame using this information and QSI. Following this, it broadcasts the selected modulation order along with the number of allocated RBs on the reserved subcarriers. As there are 8 possible modulation orders, this means an extra 3 bits are required for overhead. Just as in the centralized approach explained in Section 4.3.3, after  receiving this information tilesets reconfigure their transmission for the next frame in $T_{reconfig}$ symbols. Additionally, tilesets also tune their transmission power for the next frame, and reconfigure their receiver for proper decoding subcarriers from each tileset with different modulation orders. This procedure is illustrated in Fig. 7.4.  

\begin{figure}[htbp]
  \centering
    \includegraphics[width=1.0\textwidth]{./Chapter6_Figures/Chapter6_08.pdf}
    \rule{35em}{0.5pt}
  \caption[Centralized RB allocation and modulation order selection in WiNoCoD]{Framed structure of centralized RB allocation and modulation order selection in WiNoCoD} 
  \label{fig:Electron}
\end{figure}

\section{Maximum Delay Bounded Scheduler}

We have mentioned the possibility of saving energy, by sending data at lower rates, hence more delayed. One shall formulate this compromise elegantly in order to exploit the trade-off efficiently. One method to approach this aspect is to setting a maximum delay bound on every generated packet in the system, guaranteeing them to depart the queues in less than $D_{max}$ time, and choosing the transmission rate every instant, such that the average power is minimized. Authors in \cite{khojastepour2003power} has investigated this subject for single user and single Gaussian channel case with slotted transmission. They have shown that an optimal rate scheduler of this type is actually a \textit{Low Pass Filter}. This is intuitive as main idea of energy saving is to lower the instantaneous output rate compared to instantaneous input rate, thus lowering transmission power at the expense of delay. The dual of this perspective is the smoothing of output rate by a time averager, a linear low-pass filter. Authors have assumed arbitrary distribution of input statistics, therefore bursty arrivals. In case of linear time-invariant scheduler, where coefficients of scheduler are constant, not depending on time, they have proved that filter has a length of $D_{max}$ and all coefficients are equal to $1/D_{max}$, regardless of the statistics of input traffic. In other words, for each number of packets that arrived at time t, the optimal allocation of rate is to divide the rate equally in this slot and further $D_{max}-1$ slots. This gives the minimum transmission power for time-invariant case due to convex relation between power and rate. For instance, when a message of 6 packets arrive and we set a delay bound of 6 slots, optimal rate allocation is to serve 1 packet on current and next 5 slots, because power is an exponential function of rate. The formulation of this optimal scheduler, including also all inputs arrived up to $D_{max}-1$ slots before, is as follows :

\begin{align}
R_{t} = { {X_{t} +  X_{t-1} + .. + X_{t-(D_{max}-1)}}\over D_{max}} 
\end{align}

An example of this scheduler is shown in Fig. 7.5, where theoretically it can save up to 25 times of energy, by delaying the transmission by 5 slots, in this specific case. Further in \cite{khojastepour2004delay}, authors extend the case to linear time-varying filters, proving that they may improve the performance. However, in limited computational constraints of on-chip environment, we evaluate the possibility of adapting this basic optimal time-invariant scheduler to WiNoCoD, where there is no single channel per tileset, but changing number of multiple channels per frame. We believe, this adaptation not only improves the energy efficiency of WiNoCoD, but also extends the delay bounded optimal scheduler for all generic communication systems with varying number of channels.


\begin{figure}[htbp]
  \centering
    \includegraphics[width=0.65\textwidth]{./Chapter6_Figures/Chapter6_03.pdf}
    \rule{35em}{0.5pt}
  \caption[Maximum delay bounded rate scheduler example]{Maximum delay bounded rate scheduler can save substantial amount of power, as shown in this example} 
  \label{fig:Electron}
\end{figure}

\subsection{Extension of Delay Bounded Scheduler to Multiple Channels}

In a scenario, where a transmitter can have more than one channel, -\textit{say where each of them can serve one packet with the lowest modulation order}-, the aforementioned scheduler is not valid. This roots from the fact that, transmitter can serve more than the already scheduled rate for current symbol as it may have larger number of channels, thus optimal scheduling should be updated such that lowering rate in next slots, while ensuring serving all data in $D_{max}$ slots. Let us explain this in a basic example which is illustrated in Fig. 7.6. Say for an example that 8 packets arrive at a transmitter's  queue on a slot $t$ and maximum delay bound is 4 slots. Also, say that this transmitter has 1 channel on current symbol and 3,0,2 channels in next 3 slots, respectively. Note that each channel can serve 1 packet with the lowest modulation order. According to the maximum delay bounded optimal scheduling explained above, we should choose the lowest modulation order on every slot, such that allocating 8/4 = 2 packets/slot rate on each slot. In this case, we should choose the second modulation order (i.e. QPSK, 2 packets per channel) on the first slot, as we have 1 channel. According to equation, we should choose the lowest modulation order on second slot, as we should schedule 2 packets/slot and we have 3 channels, we can serve 3 packets on this slot, even with the lowest modulation order. Note that, we have served more packets than the scheduled 2 packets/slot. On third slot, we have no channels, therefore we cannot transmit any packets. And finally, on last slot, we have 2 channels and we should choose the lowest modulation order to schedule 2 packets/slot. However, note that a packet left in the queue which violates the maximum delay bound. 


\begin{figure}[htbp]
  \centering
    \includegraphics[width=1.0\textwidth]{./Chapter6_Figures/Chapter6_05.pdf}
    \rule{35em}{0.5pt}
  \caption[Standard low-pass filter equation is not valid for the multichannel case]{Standard low-pass filter equation is not valid for the multichannel case.} 
  \label{fig:Electron}
\end{figure}

Hence, this encouraged us to re-formulate the scheduler for changing number of channels. Another approach to look to this problem is not scheduling according to arriving number of packets in previous $D_{max}$ slots, but according to current number of packets with different deadlines (i.e. \textit{remaining time for the packet till violating delay bound}). Therefore, system needs to profile all packets in the queue, according to their deadlines. As, main idea of the optimal time-invariant scheduler is to divide the total transmission uniformly in further $D_{max}$ slots, we may think this as to allocate a total rate on current symbol, summing the different rates allocated to packets with different deadlines, dividing the number of packets to their deadline. This is formulated as follows, where $Q_{\tau}$ symbolizes the number of packets with a deadline of $\tau$ on slot t : 

\begin{align}
R_{t} =  {Q_{1} + {Q_{2}\over {2}} + .. + {Q_{\tau}\over {\tau}} + .. + {Q_{D_{max}} \over {D_{max}}}} 
\end{align}

The duality of this new formula is illustrated in Fig. 7.7, with the previous example. Note how this basic modification is valid under multiple channel case. 

\begin{figure}[htbp]
  \centering
    \includegraphics[width=1.0\textwidth]{./Chapter6_Figures/Chapter6_04.pdf}
    \rule{35em}{0.5pt}
  \caption[The proposed dual equation for maximum delay bounded scheduler]{The proposed dual equation for maximum delay bounded scheduler using instantaneous queue length} 
  \label{fig:Electron}
\end{figure}

\subsection{Maximum Delay Bounded Scheduler for WiNoCoD}

Using the principle proposed in the previous section, we design a maximum delay bounded scheduler for WiNoCoD, with a decentralized approach. Upon receiving QSI on the first symbol of a frame, tilesets run the conventional bandwidth allocation algorithm to determine their RBs to use in next frame, as previously explained pipelined fashion. Then, using its own QSI and newly allocated number of RBs, each tileset also determines its modulation order based on the maximum delay bounded rate scheduling principle and broadcasts it on the last symbol of the frame. 

As scheduling is done in frames, setting delay bounds to packets can only be effectuated in terms of frames. In other words, a packet can be guaranteed to have a maximum delay as an integer multiple of symbols in a frame. Furthermore, as new modulation order and bandwidth allocation is activated in next frame and based on the QSI broadcasted on the first symbol of current frame, we can only guarantee a minimum value for the maximum delay of 3 frames. In order to better explain this situation, we depict it in Fig. 7.8 with an example. The currently broadcasted QSI may include a packet arrived in any symbol in last frame. Therefore, the QSI may be contributed by a packet arrived $T$ symbols before. And on the other hand, as the new rate is scheduled for the next frame and all the packets are guaranteed to be served by the end of this frame, we can only guarantee the service of a packet with a maximum bound of 3 frames length.    





\begin{figure}[htbp]
  \centering
    \includegraphics[width=1.0\textwidth]{./Chapter6_Figures/Chapter6_09.pdf}
    \rule{35em}{0.5pt}
  \caption[With the proposed scheduler, the delay of packets can only be guaranteed to have a desired maximum bound with a 2 frame length addition.]{With the proposed scheduler, the delay of packets can only be guaranteed to have a desired maximum bound with a 2 frame length addition.} 
  \label{fig:Electron}
\end{figure}

\begin{figure}[htbp]
  \centering
    \includegraphics[width=1.0\textwidth]{./Chapter6_Figures/flowchart_maxdelay.pdf}
    \rule{35em}{0.5pt}
  \caption[Flow chart of the maximum delay bounded scheduler with distributed subcarrier allocation.]{Flow chart of the maximum delay bounded scheduler with distributed subcarrier allocation.} 
  \label{fig:Electron}
\end{figure}

\begin{figure*}[htbp]
  \centering
  \subcaptionbox{Non-Uniform Poisson}[0.75\linewidth][c]{%
    \includegraphics[width=0.75\linewidth]{./Chapter6_Figures/mods_4pkt_poiss.eps}}
  \subcaptionbox{Non-Uniform DPBPP (H=0.9)}[0.75\linewidth][c]{%
    \includegraphics[width=0.75\linewidth]
{./Chapter6_Figures/mods_4pkt_dpbpp.eps}}
 \caption{Packet delay exceeding probability graphs for non-uniform Poisson and DPBPP traffic with injection rate of 4 packets/symbol for a static modulation order system for different utilized modulation orders (EQPS($\alpha = 0.95$), time direction allocation, T=8 symbols)}
\end{figure*}

\begin{figure}[htbp]
  \centering
    \includegraphics[width=1.0\textwidth]{./Chapter6_Figures/mods.eps}
    \rule{35em}{0.5pt}
  \caption[The average transmission power increases drastically, if higher modulation orders are used constantly.]{The average transmission power increases drastically, if higher modulation orders are used constantly.} 
  \label{fig:Electron}
\end{figure}
\subsubsection{Experimental Evaluation}


\begin{figure*}[htbp]
  \centering
  \subcaptionbox{Packet delay exceeding probability graph. For each tried maximum delay bound (in frames + 2 frame length addition).}[0.75\linewidth][c]{%
    \includegraphics[width=0.75\linewidth]{./Chapter6_Figures/MaxDelayBound_InjRate4_NonUniPoiss2.eps}}
  \subcaptionbox{The resulting average power in terms of transmission power required for a single RB with BPSK, for each tried maximum delay bound in frames. (Actual delay bound is 2 frames more)}[0.75\linewidth][c]{%
    \includegraphics[width=0.75\linewidth]
{./Chapter6_Figures/MaxDelayBound_InjRate4_NonUniPoiss_AvgPowers.eps}}
 \caption{Decentralized maximum delay bounded scheduler performance under non-uniform Poisson traffic with injection rate of 4 packets/symbol (EQPS($\alpha = 0.95$), time direction allocation, T=8 symbols)}
\end{figure*}

We have tested the proposed maximum delay bounded scheduler with the aforementioned decentralized context, by using Expected Queue Proportional Scheduling (EQPS) explained in 5.2.1 with allocation in time direction. The main reason behind that it allows the allocation of at least 1 RB to every tileset which is likely not to be idle. If a tileset is not allocated any frequency resource in a frame, setting the modulation order, therefore the necessary instantaneous rate is not possible. In addition, we have shown that EQPS provides remarkably good performance under high injection rates. 8 modulation orders from BPSK (1 bit/subcarrier) to 256-QAM (8 bits/subcarrier) are utilized, as mentioned previously. However, we remind that any other kind of subcarrier arbitration algorithm can be implemented on this decentralized maximum delay bounded modulation order selection framework. Due to limited space, only a few experiments are demonstrated. Note that, this time the lowest modulation order is BPSK. We have used QPSK as default in previous chapters, thus this means the capacity of the system is halved if only BPSK is utilized. In addition, QSI and modulation order signaling is always done in BPSK for minimizing probability of error. Therefore, the required number of RBs allocated for signaling is two times more in this case. For total injection rate values of 4, 6 and 8 packets/symbol, the simulation for a frame length of 8 symbols is executed under non-uniform Poisson and DPBPP traffic with H=0.9, for different maximum delay bounds. 



\begin{figure*}[htbp]
  \centering
  \subcaptionbox{Packet delay exceeding probability graph. For each tried maximum delay bound (in frames + 2 frame length addition).}[0.75\linewidth][c]{%
    \includegraphics[width=0.75\linewidth]{./Chapter6_Figures/MaxDelayBound_InjRate4_DPBPP.eps}}
  \subcaptionbox{The resulting average power in terms of transmission power required for a single RB with BPSK, for each tried maximum delay bound in frames. (Actual delay bound is 2 frames more)}[0.75\linewidth][c]{%
    \includegraphics[width=0.75\linewidth]
{./Chapter6_Figures/MaxDelayBound_InjRate4_DPBPP_AvgPowers.eps}}
 \caption{Decentralized maximum delay bounded scheduler performance non-uniform DPBPP traffic (H=0.9) with injection rate of 4 packets/symbol (EQPS($\alpha = 0.95$), time direction allocation, T=8 symbols)}
\end{figure*}

\begin{figure*}[htbp]
  \centering
  \subcaptionbox{Non-Uniform Poisson}[0.75\linewidth][c]{%
    \includegraphics[width=0.75\linewidth]{./Chapter6_Figures/mods_6pkt_poiss.eps}}
  \subcaptionbox{Non-Uniform DPBPP (H=0.9)}[0.75\linewidth][c]{%
    \includegraphics[width=0.75\linewidth]
{./Chapter6_Figures/mods_6pkt_dpbpp.eps}}
 \caption{Packet delay exceeding probability graphs for non-uniform Poisson and DPBPP traffic with injection rate of 6 packets/symbol for a static modulation order system for different utilized modulation orders (EQPS($\alpha = 0.95$), time direction allocation, T=8 symbols)}
\end{figure*}

%
\begin{figure*}[htbp]
  \centering
  \subcaptionbox{Packet delay exceeding probability graph. For each tried maximum delay bound (in frames + 2 frame length addition).}[0.75\linewidth][c]{%
    \includegraphics[width=0.75\linewidth]{./Chapter6_Figures/MaxDelayBound_InjRate6_NonUniPoiss2.eps}}
  \subcaptionbox{The resulting average power in terms of transmission power required for a single RB with BPSK, for each tried maximum delay bound in frames. (Actual delay bound is 2 frames more)}[0.75\linewidth][c]{%
    \includegraphics[width=0.75\linewidth]
{./Chapter6_Figures/MaxDelayBound_InjRate6_NonUniPoiss_AvgPowers.eps}}
 \caption{Decentralized maximum delay bounded scheduler performance under non-uniform Poisson traffic with injection rate of 6 packets/symbol (EQPS($\alpha = 0.95$), time direction allocation, T=8 symbols)}
\end{figure*}


\begin{figure*}[htbp]
  \centering
  \subcaptionbox{Packet delay exceeding probability graph. For each tried maximum delay bound (in frames + 2 frame length addition).}[0.75\linewidth][c]{%
    \includegraphics[width=0.75\linewidth]{./Chapter6_Figures/MaxDelayBound_InjRate6_DPBPP.eps}}
  \subcaptionbox{The resulting average power in terms of transmission power required for a single RB with BPSK, for each tried maximum delay bound in frames. (Actual delay bound is 2 frames more)}[0.75\linewidth][c]{%
    \includegraphics[width=0.75\linewidth]
{./Chapter6_Figures/MaxDelayBound_InjRate6_DPBPP_AvgPowers.eps}}
 \caption{Decentralized maximum delay bounded scheduler performance under under non-uniform DPBPP traffic (H=0.9) with injection rate of 6 packets/symbol (EQPS($\alpha = 0.95$), time direction allocation, T=8 symbols)}
\end{figure*}

\textbf{\textit{Injection Rate = 4 packets/symbol}}

Before testing our algorithm for the injection rate of 4 packets/symbol non-uniform Poisson and DPBPP traffic, in Fig. 7.10(a) and in Fig. 7.10(b), we provide the delay exceeding probability graphs for the framed EQPS allocation with static modulation order under these scenarios. The reference static modulation order for modulation orders from BPSK to 256-QAM uses the same frame length and bandwidth allocation mechanism, and consitutes a reference for the proposed dynamic modulation order. Remind that, utilizing higher modulation orders constantly gives much more lower delay exceeding probabilities at the expense of exponentially higher modulation orders. The resulting average power in terms of power required for BPSK with constant modulation order utilization is shown in Fig. 7.11.




Fig. 7.12(a) shows the probability of a packet exceeding a given delay for each tried maximum delay bound with the proposed scheduler under non-uniform Poisson traffic with an injection rate of 4 packets/symbol. As mentioned previously, the delay bounds can be quantified in our algorithm in terms of frame lengths and 2 frames length shall be added to this initially given bound due to the structure of the algorithm. For instance, in this case we have assumed a frame length of 8 symbols; therefore one can start to guarantee a minimum value for maximum delay bound as 3 frames (24 symbols). By inspecting Fig. 7.12(a), we see that except for the 1 (1+2) frame length (24 symbols), all the given delay bounds have  been validated (For a delay bound of 24 symbols, probability of exceeding 24 symbols is around 0.0003). In addition, after 24 symbols bound, the algorithm operated with 2 (2+2) frame length (32 symbols) perform even better compared to 1 (1+2) frame length bound. This is because, when we impose a delay bound of 1 frame (where it is a single length filter), most of the time the algorithm can not perform coherently, as it can not sustain the necessary instantaneous rate with the highest modulation order of 256-QAM. Also, frame structure causing outdated information on queue lengths, makes tight delay bounds almost impossible.  


In Fig. 7.12(b), we see the resulting average power expenditure for 8 different delay bounds (from 24 symbols/1+2 frames to 80 symbols/8+2 frames). We can see that, as we are increasing the maximum delay bound, we can decrease the consumed avarage power. For a 1 (1+2) frame length (24 symbols) maximum delay bound, the resulting average power is slightly above 10 units (in terms of minimum transmission power required for a single RB (32 subcarriers) coded with BPSK); whereas for a delay bound of 4 (4+2) frame length (48 symbols), the resulting average power is around 2 power units. This means, by increasing average delay bound 2 times for this scenario, we can decrease the total energy expenditure by 5 fold. For higher delay bounds, we observe that there are no more substantial power gains. This scheme is predictable and consistent with the previously explained Pareto like delay-power relation.

Then, we evaluate our algorithm for the same injection rate of 4 packets/symbol, but for realistic non-uniform DPBPP traffic with H=0.9 as in Fig. 7.13. First of all, we see that intended delay bounds of 1 (1+2) frame (24 symbols), 2 (2+2) frames (32 symbols), 3 (3+2) frames (40 symbols) and 4 (4+2) frames (48 symbols) could  not have been achieved, with a suitable probability of error. We observe that the probability of exceeding these delay bounds are 0.0413, 0.0067, 0.0025, 0.0012 respectively (See Table 7.1). We also observe that, delay bound of 1 frame is even performing worse than a delay bound of 2 frames. This inconsistency is similar to the previous non-uniform Poisson case, but even worse. This is due to the more temporal burstiness of DPBPP traffic. Our algorithm can not provide the instantaneous peaks in traffic demands, as the highest modulation order is limited and pipelined/framed structure does not allow for efficient rate allocation, especially under rapidly fluctuating traffic. At this point, it is also important to note that for delay bounds larger than 60 symbols, all algorithms (except for delay bound with 1 frame), converge to nearly same exceeding probabilities. We observe that, even larger delay bounded algorithms provide lesser probabilities of exceeding delays larger than 60 symbols. This is also an unexpected and erroneous result. One of the reasons behind this is that even for delay bounds with several frames, the packets experiencing delays larger than 60 symbols is a relatively rare case. Due to highly bursty traffic, queue dynamics would not evolve consistently, with dynamic rate allocation. However, we can still claim that the proposed algorithm provides a maximum delay bounded energy minimization with a reliability up to a degree.  



In Fig. 7.13(b), we see the resulting average power expenditure for 8 different delay bounds (from 24 symbols/1+2 frames to 80 symbols/8+2 frames). First noteworthy result from this figure, is that 1 frame length maximum delay bound has even consumed less power, compared to 2 frame length maximum delay bound. This proves an invalidity for the performance of the algorithm with 1 frame length bound. This errenous result is also highly consistent with the observation we made for Fig. 7.13(a), for the probability of exceeding a packet delay; that it looks like the algorithm with 2 frame length bound has chosen even higher rates compared to 1 frame length bound. However, we observe that, as we are increasing the maximum delay bound above 2 frames, we can decrease the consumed average power substantially. For instance, by decreasing maximum delay bound from 2 (2+2) frames to  8 (8+2) frames we can save up to 8 times of energy. As a last point, also note that with bursty traffic, the average power expenditure is significantly higher for all different delay bounds compared to Poisson traffic; as higher modulation orders are selected much more frequently under fluctuating traffic demands. 

\begin{figure*}[htbp]
  \centering
  \subcaptionbox{Non-Uniform Poisson}[0.75\linewidth][c]{%
    \includegraphics[width=0.75\linewidth]{./Chapter6_Figures/mods_8pkt_poiss.eps}}
  \subcaptionbox{Non-Uniform DPBPP (H=0.9)}[0.75\linewidth][c]{%
    \includegraphics[width=0.75\linewidth]
{./Chapter6_Figures/mods_8pkt_dpbpp.eps}}
 \caption{Packet delay exceeding probability graphs for non-uniform Poisson and DPBPP traffic with injection rate of 8 packets/symbol for a static modulation order system for different utilized modulation orders (EQPS($\alpha = 0.95$), time direction allocation, T=8 symbols)}
\end{figure*}


\begin{figure*}[htbp]
  \centering
  \subcaptionbox{Packet delay exceeding probability graph. For each tried maximum delay bound (in frames + 2 frame length addition).}[0.75\linewidth][c]{%
    \includegraphics[width=0.75\linewidth]{./Chapter6_Figures/MaxDelayBound_InjRate8_NonUniPoiss.eps}}
  \subcaptionbox{The resulting average power in terms of transmission power required for a single RB with BPSK, for each tried maximum delay bound in frames. (Actual delay bound is 2 frames more)}[0.75\linewidth][c]{%
    \includegraphics[width=0.75\linewidth]
{./Chapter6_Figures/MaxDelayBound_InjRate8_NonUniPoiss_AvgPowers.eps}}
 \caption{Decentralized maximum delay bounded scheduler performance under non-uniform Poisson traffic with injection rate of 8 packets/symbol (EQPS($\alpha = 0.95$), time direction allocation, T=8 symbols)}
\end{figure*}

\begin{figure*}[htbp]
  \centering
  \subcaptionbox{Packet delay exceeding probability graph. For each tried maximum delay bound (in frames + 2 frame length addition).}[0.75\linewidth][c]{%
    \includegraphics[width=0.75\linewidth]{./Chapter6_Figures/MaxDelayBound_InjRate8_DPBPP.eps}}
  \subcaptionbox{The resulting average power in terms of transmission power required for a single RB with BPSK, for each tried maximum delay bound in frames. (Actual delay bound is 2 frames more)}[0.75\linewidth][c]{%
    \includegraphics[width=0.75\linewidth]
{./Chapter6_Figures/MaxDelayBound_InjRate8_DPBPP_AvgPowers.eps}}
 \caption{Decentralized maximum delay bounded scheduler performance under under non-uniform DPBPP traffic (H=0.9) with injection rate of 8 packets/symbol (EQPS($\alpha = 0.95$), time direction allocation, T=8 symbols)}
\end{figure*}


\textbf{\textit{Injection Rate = 6 packets/symbol}}

We perform the same experiments by increasing total injection rate to 6 packets/symbol. Fig. 7.14 shows the packet delay exceeding probabilities for the static modulation EQPS order case for different modulation orders under if statically used. Fig 7.15(a) shows the propbability of delay exceeding for different delay bounds for non-uniform Poisson and DPBPP traffic, under 6 packets/symbol total injection rate. Similarly, only the bound of algorithm with delay bound of 1 frame is violated. We observe from Fig. 7.15(b) that, by decreasing maximum delay bound from 1 (1+2) frames to  4 (4+2) frames we can save up to 5 times of energy. And next, in Fig. 7.15(a), exceeding probabilities with different delay bounds for non-uniform DPBPP traffic is evaluated. This time, all the tried delay bounds are violated and have probabilities of violation between 0.1 and 0.01. As expected, when the average traffic load and temporal burstiness increase, performance of the algorithm degrades. By increasing delay bound from 1 (1+2) frame to 8 (8+2) frames, up to 7 times of average power can be saved.  


 \begin{table}[ht]
\centering
\caption{Probability of exceeding the intended delay bound of the algorithm for different traffic intensities and models.}
\label{Probability of exceeding the intended delay bound of the algorithm for different traffic intensities and models.}
\begin{adjustbox}{width=1\textwidth}
\begin{tabular}{l|l|lllllll}
\cline{2-2}
                                                                                                                             & \textit{\textbf{\begin{tabular}[c]{@{}l@{}}Given \\ Delay \\ Bound\end{tabular}}} &                                                                                                  &                                                                                                  &                                                                                                  &                                                                                                 &                                                                                                  &                                                                                                  &                                                                                                  \\ \hline
\multicolumn{1}{|l|}{\textit{\textbf{\begin{tabular}[c]{@{}l@{}}Probability \\ of \\ Exceding \\ Delay Bound\end{tabular}}}} & \textbf{\begin{tabular}[c]{@{}l@{}}1+2 Frames \\ (24 symbols)\end{tabular}}       & \multicolumn{1}{l|}{\textbf{\begin{tabular}[c]{@{}l@{}}2+2 Frames \\ (32 symbols)\end{tabular}}} & \multicolumn{1}{l|}{\textbf{\begin{tabular}[c]{@{}l@{}}3+2 Frames \\ (40 symbols)\end{tabular}}} & \multicolumn{1}{l|}{\textbf{\begin{tabular}[c]{@{}l@{}}4+2 Frames \\ (48 symbols)\end{tabular}}} & \multicolumn{1}{l|}{\textbf{\begin{tabular}[c]{@{}l@{}}5+2 Frames\\ (56 symbols)\end{tabular}}} & \multicolumn{1}{l|}{\textbf{\begin{tabular}[c]{@{}l@{}}6+2 Frames \\ (64 symbols)\end{tabular}}} & \multicolumn{1}{l|}{\textbf{\begin{tabular}[c]{@{}l@{}}7+2 Frames \\ (72 symbols)\end{tabular}}} & \multicolumn{1}{l|}{\textbf{\begin{tabular}[c]{@{}l@{}}8+2 Frames \\ (80 symbols)\end{tabular}}} \\ \hline
\multicolumn{1}{|l|}{\textbf{\begin{tabular}[c]{@{}l@{}}Inj. Rate = 4 \\ pkts/sym \\ (Non-Uni. Poiss.)\end{tabular}}}        & 0.0003                                                                            & \multicolumn{1}{l|}{0}                                                                           & \multicolumn{1}{l|}{0}                                                                           & \multicolumn{1}{l|}{0}                                                                           & \multicolumn{1}{l|}{0}                                                                          & \multicolumn{1}{l|}{0}                                                                           & \multicolumn{1}{l|}{0}                                                                           & \multicolumn{1}{l|}{0}                                                                           \\ \hline
\multicolumn{1}{|l|}{\textbf{\begin{tabular}[c]{@{}l@{}}Inj. Rate = 4 \\ pkts/sym \\ (DPBPP)\end{tabular}}}                  & 0.0413                                                                            & \multicolumn{1}{l|}{0.0067}                                                                      & \multicolumn{1}{l|}{0.0025}                                                                      & \multicolumn{1}{l|}{0.0012}                                                                      & \multicolumn{1}{l|}{0.0005}                                                                     & \multicolumn{1}{l|}{0.0002}                                                                      & \multicolumn{1}{l|}{0.0001}                                                                      & \multicolumn{1}{l|}{0}                                                                           \\ \hline
\multicolumn{1}{|l|}{\textbf{\begin{tabular}[c]{@{}l@{}}Inj. Rate = 6 \\ pkts/sym  \\ (Non-Uni. Poiss.)\end{tabular}}}       & 0.0017                                                                            & \multicolumn{1}{l|}{0}                                                                           & \multicolumn{1}{l|}{0}                                                                           & \multicolumn{1}{l|}{0}                                                                           & \multicolumn{1}{l|}{0}                                                                          & \multicolumn{1}{l|}{0}                                                                           & \multicolumn{1}{l|}{0}                                                                           & \multicolumn{1}{l|}{0}                                                                           \\ \hline
\multicolumn{1}{|l|}{\textbf{\begin{tabular}[c]{@{}l@{}}Inj. Rate = 6 \\ pkts/sym  \\ (DPBPP)\end{tabular}}}                 & 0.0958                                                                            & \multicolumn{1}{l|}{0.0176}                                                                      & \multicolumn{1}{l|}{0.008}                                                                       & \multicolumn{1}{l|}{0.0053}                                                                      & \multicolumn{1}{l|}{0.0039}                                                                     & \multicolumn{1}{l|}{0.0024}                                                                      & \multicolumn{1}{l|}{0.0015}                                                                      & \multicolumn{1}{l|}{0.0007}                                                                      \\ \hline
\multicolumn{1}{|l|}{\textbf{\begin{tabular}[c]{@{}l@{}}Inj. Rate = 8 \\ pkts/sym  \\ (Non-Uni. Poiss.)\end{tabular}}}       & 0.0069                                                                            & \multicolumn{1}{l|}{0.0001}                                                                      & \multicolumn{1}{l|}{0}                                                                           & \multicolumn{1}{l|}{0}                                                                           & \multicolumn{1}{l|}{0}                                                                          & \multicolumn{1}{l|}{0}                                                                           & \multicolumn{1}{l|}{0}                                                                           & \multicolumn{1}{l|}{0}                                                                           \\ \hline
\multicolumn{1}{|l|}{\textbf{\begin{tabular}[c]{@{}l@{}}Inj. Rate = 8 \\ pkts/sym \\ (DPBPP)\end{tabular}}}                  & 0.2409                                                                            & \multicolumn{1}{l|}{0.0473}                                                                      & \multicolumn{1}{l|}{0.0278}                                                                      & \multicolumn{1}{l|}{0.0278}                                                                      & \multicolumn{1}{l|}{0.0339}                                                                     & \multicolumn{1}{l|}{0.0446}                                                                      & \multicolumn{1}{l|}{0.0441}                                                                      & \multicolumn{1}{l|}{0.0321}                                                                      \\ \hline
\end{tabular}
\end{adjustbox}
\end{table}

 \begin{table}[ht]
\centering
\caption{Achieved average power with the intended delay bound in terms of power required for 1 RB with BPSK for different traffic intensities and models.}
\label{Achieved average power with the intended delay bound in terms of power required for 1 RB with BPSK for different traffic intensities and models.}
\begin{adjustbox}{width=1\textwidth}
\begin{tabular}{l|l|lllllll}
\cline{2-2}
                                                                                                                       & \textit{\textbf{\begin{tabular}[c]{@{}l@{}}Given \\ Delay \\ Bound\end{tabular}}} &                                                                                                  &                                                                                                  &                                                                                                  &                                                                                                 &                                                                                                  &                                                                                                  &                                                                                                  \\ \hline
\multicolumn{1}{|l|}{\textit{\textbf{\begin{tabular}[c]{@{}l@{}}Average\\ Power\end{tabular}}}}                        & \textbf{\begin{tabular}[c]{@{}l@{}}1+2 Frames \\ (24 symbols)\end{tabular}}       & \multicolumn{1}{l|}{\textbf{\begin{tabular}[c]{@{}l@{}}2+2 Frames \\ (32 symbols)\end{tabular}}} & \multicolumn{1}{l|}{\textbf{\begin{tabular}[c]{@{}l@{}}3+2 Frames \\ (40 symbols)\end{tabular}}} & \multicolumn{1}{l|}{\textbf{\begin{tabular}[c]{@{}l@{}}4+2 Frames \\ (48 symbols)\end{tabular}}} & \multicolumn{1}{l|}{\textbf{\begin{tabular}[c]{@{}l@{}}5+2 Frames\\ (56 symbols)\end{tabular}}} & \multicolumn{1}{l|}{\textbf{\begin{tabular}[c]{@{}l@{}}6+2 Frames \\ (64 symbols)\end{tabular}}} & \multicolumn{1}{l|}{\textbf{\begin{tabular}[c]{@{}l@{}}7+2 Frames \\ (72 symbols)\end{tabular}}} & \multicolumn{1}{l|}{\textbf{\begin{tabular}[c]{@{}l@{}}8+2 Frames \\ (80 symbols)\end{tabular}}} \\ \hline
\multicolumn{1}{|l|}{\textbf{\begin{tabular}[c]{@{}l@{}}Inj. Rate = 4 \\ pkts/sym \\ (Non-Uni. Poiss.)\end{tabular}}}  & 10.52                                                                             & \multicolumn{1}{l|}{8.72}                                                                        & \multicolumn{1}{l|}{4.09}                                                                        & \multicolumn{1}{l|}{1.97}                                                                        & \multicolumn{1}{l|}{1.46}                                                                       & \multicolumn{1}{l|}{1.27}                                                                        & \multicolumn{1}{l|}{1.17}                                                                        & \multicolumn{1}{l|}{1.11}                                                                        \\ \hline
\multicolumn{1}{|l|}{\textbf{\begin{tabular}[c]{@{}l@{}}Inj. Rate = 4 \\ pkts/sym \\ (DPBPP)\end{tabular}}}            & 36.77                                                                             & \multicolumn{1}{l|}{38.51}                                                                       & \multicolumn{1}{l|}{29.88}                                                                       & \multicolumn{1}{l|}{18.97}                                                                       & \multicolumn{1}{l|}{11.85}                                                                      & \multicolumn{1}{l|}{7.74}                                                                        & \multicolumn{1}{l|}{5.4}                                                                         & \multicolumn{1}{l|}{3.97}                                                                        \\ \hline
\multicolumn{1}{|l|}{\textbf{\begin{tabular}[c]{@{}l@{}}Inj. Rate = 6 \\ pkts/sym  \\ (Non-Uni. Poiss.)\end{tabular}}} & 28.94                                                                             & \multicolumn{1}{l|}{27.58}                                                                       & \multicolumn{1}{l|}{15.25}                                                                       & \multicolumn{1}{l|}{6.67}                                                                        & \multicolumn{1}{l|}{4.4}                                                                        & \multicolumn{1}{l|}{3.56}                                                                        & \multicolumn{1}{l|}{3.17}                                                                        & \multicolumn{1}{l|}{2.93}                                                                        \\ \hline
\multicolumn{1}{|l|}{\textbf{\begin{tabular}[c]{@{}l@{}}Inj. Rate = 6 \\ pkts/sym  \\ (DPBPP)\end{tabular}}}           & 64.58                                                                             & \multicolumn{1}{l|}{69.6}                                                                        & \multicolumn{1}{l|}{56.73}                                                                       & \multicolumn{1}{l|}{38.89}                                                                       & \multicolumn{1}{l|}{26.57}                                                                      & \multicolumn{1}{l|}{18.78}                                                                       & \multicolumn{1}{l|}{13.88}                                                                       & \multicolumn{1}{l|}{10.78}                                                                       \\ \hline
\multicolumn{1}{|l|}{\textbf{\begin{tabular}[c]{@{}l@{}}Inj. Rate = 8 \\ pkts/sym  \\ (Non-Uni. Poiss.)\end{tabular}}} & 58.25                                                                             & \multicolumn{1}{l|}{59.56}                                                                       & \multicolumn{1}{l|}{38.49}                                                                       & \multicolumn{1}{l|}{18.95}                                                                       & \multicolumn{1}{l|}{12.37}                                                                      & \multicolumn{1}{l|}{9.65}                                                                        & \multicolumn{1}{l|}{8.32}                                                                        & \multicolumn{1}{l|}{7.55}                                                                        \\ \hline
\multicolumn{1}{|l|}{\textbf{\begin{tabular}[c]{@{}l@{}}Inj. Rate = 8 \\ pkts/sym \\ (DPBPP)\end{tabular}}}            & 98.11                                                                             & \multicolumn{1}{l|}{109.48}                                                                      & \multicolumn{1}{l|}{93.49}                                                                       & \multicolumn{1}{l|}{68.53}                                                                       & \multicolumn{1}{l|}{50.72}                                                                      & \multicolumn{1}{l|}{39.01}                                                                       & \multicolumn{1}{l|}{30.99}                                                                       & \multicolumn{1}{l|}{25.37}                                                                       \\ \hline
\end{tabular}
\end{adjustbox}
\end{table}

\textbf{\textit{Injection Rate = 8 packets/symbol}}


And lastly, we evaluate the proposed maximum delay bounded modulation order selection and bandwidth allocation algorithm for a total injection rate of 8 packets/symbol. Fig. 7.17 shows the packet delay exceeding probabilities for the static modulation orders EQPS order case for different modulation orders under. Fig 7.18(a) shows the probability of delay exceeding for different delay bounds for non-uniform Poisson and DPBPP traffic, under 8 packets/symbol total injection rate. First of all, we highlight this important point : theoretically, our interconnect should support a total injection rate up to 45 packets/symbol with highest modulation order of 256-QAM. However, we have seen that proposed maximum delay bounded algorithm fails for every given delay bound for total injection rates higher approximately 8 packets/symbol. This stems from the exact same reason behind the violations observed in previous experiments. Due to framed nature of the algorithm and limited rate, most of the time certain traffic peaks cannot be served, thus queue dynamics fail. However, we again note that, this algorithm may provide an efficient scheduling for energy expenditure minimization, especially under lower traffic loads and heterogeneity.   



In Fig. 7.18(a), delay exceeding probabilities for algorithms with different maximum delay bounds are shown, for a total injection rate of 8 packets/symbol under non-uniform Poisson traffic. For another time, we observe that the algorithm with 1 (1+2) frame length bound is far to provide the desired performance. Fig. 7.18(b), shows that we can decrease the average power by 3 times, by increasing delay bound from 2 (2+2) frames to 4 (4+2) frames.



And finally, we evaluate our algorithm under non-uniform DPBPP traffic. Fig. 7.19(a) shows that all intended delay bounds are violated with a probability of between 0.1 and 0.01, which is quite undesired. However, we can still claim, even under highly bursty traffic and higher loads, this algorithm can save power while providing not a strict delay bound, but a reasonable expectation of  maximum delay violation. Fig. 7.19(b) shows that, we can save approximately 3 times of power by reducing the maximum delay bound from 2 (2+2) frames to 7 (7+2) frames. 

  
\section{Average Delay Bounded Scheduler}


We propose to develop another modulation order scheduler for WiNoCoD, which is based on \cite{rajan2004delay}. This time the main motivation is to minimize transmission power, while setting a bound on \textit{average delay} of the packets, which is a more valuable metric of interest, as mentioned previously. \cite{rajan2004delay} presents a scheduler for a single user single AWGN channel under bursty arrivals, which gives the minimum transmission power, while setting a bound on average delay. Optimal solution includes complex processes and operations. In addition, in order to apply these stochastic optimization techniques, unrealistic assumptions on arrival traffic have to be made, such as no temporal correlation. 


The authors make a remarkable observation on the optimal scheduler that rate (modulation order) is chosen approximately proportional to natural logarithm of instantaneous queue state. They explain this behavior with the tendency of scaling rate linearly proportional to consumed power (Note that, required transmission power is proportional to exponential of scheduled rate). Then, they propose a near optimal heuristics based scheduler called \textit{log-linear scheduler}, which chooses the modulation order proportional to instantaneous queue state. To perform this, they multiply queue state before taking its natural logarithm, by a scalar $\kappa$, which is incremented by a small step size if the current estimated average delay is larger than the desired average delay bound. Similarly, if the currently estimated average delay meets the desired bound, $\kappa$ is decremented by a small step size. As $\kappa$ gets larger, the resulting logarithm would be larger, thus choosing higher rates. This way, the minimum transmission power is tried to be achieved while respecting the average delay bound. This near-optimal scheduler hugely simplifies modulation order selection procedure, as it just requires few number of basic mathematical operations. 

\subsection{Average Delay Bounded Scheduling with Centralized Approach Based on EQPS}

Using the strong efficiency of this computationally trivial tool, we would like apply it to WiNoCoD. But as previously stated, this scheduler is for the single-user, single-channel case, therefore it shall be adapted to multi-user, multi-channel nature of WiNoCoD. In contrast with the previous section, average delay bounded modulation order selection algorithm is demonstrated with the centralized paradigm. The reason behind this is first to show the ability of using fully centralized intelligence including also modulation order selection and second is the nature of average delay bounded scheduler which enables centralized computation. The algorithm is executed as explained in 7.3.2, that the each tileset broadcasts its QSI and CIU calculates the number of RB allocated to each. Then using these information, CIU also calculates the optimal modulation order for each tileset, using average delay bounded scheduler. Finally, CIU broadcasts number of allocated RBs and selected modulation order for each tileset, 2 symbols before the end of the frame. Hence, tilesets can reconfigure their transmission patterns and powers at the start of next frame. Similarly for the maximum delay bounded algorithm, EQPS ($\alpha = 0.95$) in time direction is chosen to allocate frequency RBs.

The idea of the algorithm is straightforward; as in \cite{rajan2004delay}, modulation order (thus rate) is chosen proportional to natural logarithm of the QSI for each tileset. However, different from this, a tileset shall have different number of RBs (each can serve 32 bits flit with BPSK, 2 flits with QPSK, 3 flits with 16-QAM, .. , 8 flits with 256-QAM). Hence, if there are $S$ RBs allocated to a tileset, and the chosen modulation rate can serve M flits (32 bits) per RB, the total service rate of this tileset will be $MS$ through the whole frame. Therefore, we have modified the algorithm such that CIU choose the modulation order  of tileset $i$, proportional to the natural logarithm of its QSI divided by the number of RBs allocated :


\begin{align}
 M_{i}^{t+1} = \left \lceil log \left( \frac{\kappa_{i}{\hat{Q}}_{i}^{t}}{S_{i}^{t+1}} \right) \right \rceil 
\end{align}   

Just as in \cite{rajan2004delay}, $\kappa_{i}$ is a dynamic scalar in order to achieve the desired average delay bound $D_{avg}$. CIU keeps a $\kappa_{i}$ value for each tileset and updates every frame. If currently calculated average delay ${{\hat{D}}_{i}^{t}}$ is under the delay bound it is decreased by a constant differential, $\Delta \kappa$ multiplied by the difference $({{\hat{D}}_{i}^{t}}-D_{avg})$ to tune the algorithm. This way, the chosen modulation order, therefore the transmission power can be minimized as much as possible given that we are not violating the average delay bound. If calculated instantaneous average delay sample is exceeding the $D_{avg}$, $\kappa_{i}$ is increased by $\Delta \kappa ({{\hat{D}}_{i}^{t}}-D_{avg})$. As you can see, increasing $\kappa_{i}$ values tend to choose higher modulation orders. 

So in order to update  $\kappa_{i}$, CIU has to compute instantaneous average delay sample for each tileset. As in the reference paper, \textit{Little's Law} \cite{rajan2004delay} is used to estimate average waiting time. Little's Law states that for any kind of queuing system with any kind of input process, division of average queue state by average number of packets arrive gives the average waiting time in the system. Note that CIU keeps already average arrival rates (estimated number of flits per frame) with a exponential moving averaging filter. In addition to it, another 32 Exponentially Weighted Moving Average (EWMA) filters are kept for averaging QSIs. Then division of these two values gives the moving average of the estimated average delay of the tileset :

\begin{align}
{{\hat{D}}_{i}^{t}} = \frac{{{\hat{Q}}_{i}^{t}}} {{{\hat{A}}_{i}^{t}}} 
\end{align}

Then for each 32 tilesets, $\kappa_{i}$ values are updated as follows :

\begin{align}
\kappa_{i} =\kappa_{i} + \Delta \kappa ({{\hat{D}}_{i}^{t}}-D_{avg})
\end{align}

Log-linear algorithm uses just basic mathematical operators to give a solution to complex average delay bounded power minimization problem, except the logarithm. This can be performed by a simple look-up table which decreases its latency to few cycles. And as for EQPS, division can be done by a simple reciprocal multiplication. For each tileset, CIU has to perform 1 float division (1 time division to calculate the reciprocal of ${{\hat{Q}}_{i}^{t}}$), 3 float multiplications, 1 division-by-2 (bit shifting) operation, 1 natural logarithm by look-up table and finally an upper-rounding. 

In addition to this, the computation for estimated average delay and $\kappa_{i}$ updates shall be performed. At first, as for (7.2), the 32 EWMA calculations for QSIs had to be done, which includes 2 float multiplications and 1 addition. Then 1 division is done for (7.6). Next, for $\kappa_{i}$ updates, 1 float multiplication, 1 subtraction and 1 addition is done. Note that, these computations for estimating average delay and $\kappa_{i}$ updates can be done in the period where tilesets reconfigure their transmissions (as for EWMA calculations of average arrivals). We remind parallelism can be exploited for these operations for 32 tilesets by employing multiple simple cores at CIU, which decreases required computation time further and further.

\begin{figure}[htbp]
  \centering
    \includegraphics[width=0.75\textwidth]{./Chapter6_Figures/flowchart_avgdelay.pdf}
    \rule{35em}{0.5pt}
  \caption[Flow chart of the average delay bounded scheduler algorithm executed at CIU every frame.]{Flow chart of the average delay bounded scheduler algorithm executed at CIU every frame.} 
  \label{fig:Electron}
\end{figure}

Fig. 7.20 illustrates the flow chart of the proposed average delay bounded scheduler inside the CIU, including the estimation of arriving number of flits in each tileset, estimation of instantaneous average delays and update of the $\kappa$ values. 





\subsection{Experimental Evaluation}

Now, we will test the performance of the proposed scheduler for various frame lengths and for each tried average delay bound from 1 to 20 symbols, and under different injection rates. Due to lack of space, only realistic and highly bursty non-uniform DPBPP traffic case is considered. Log-linear algorithm was shown to operate with a performance close to optimal scheduler for the single-user, single-channel case. One assumption that the authors had made, was the i.i.d. nature of the bursty arrival model. Therefore, with self-similarity we can accept to divert from optimality, however scaling transmission power linearly with increasing backlog shall still provide an effective delay-power trade-off.





\textbf{\textit{Frame Length = 4+2 symbols}}


\begin{figure}[t!]
  \centering
    \includegraphics[width=0.75\textwidth]{./Chapter6_Figures/avgDelayBound_T4_Inj8_DPBPP_v2.eps}
    \rule{35em}{0.5pt}
  \caption[Resulting average delay and average power is shown with a 2-y plot, with T =4+2 symbols, under non-uniform DPBPP traffic under an injection rate of 8 packets/symbol.]{Resulting average delay and average power is shown with a 2-y plot, with T =4+2 symbols, under non-uniform DPBPP traffic under an injection rate of 8 packets/symbol.} 
  \label{fig:Electron}
\end{figure}

\begin{figure}[t!]
  \centering
    \includegraphics[width=0.75\textwidth]{./Chapter6_Figures/avgDelayBound_T4_Inj12_DPBPP.eps}
    \rule{35em}{0.5pt}
  \caption[Resulting average delay and average power is shown with a 2-y plot, with T =4+2 symbols, under non-uniform DPBPP traffic under an injection rate of 8 packets/symbol.]{Resulting average delay and average power is shown with a 2-y plot, with T =4+2 symbols, under non-uniform DPBPP traffic under an injection rate of 12 packets/symbol.} 
  \label{fig:Electron}
\end{figure}

\begin{figure}[t!]
  \centering
    \includegraphics[width=0.75\textwidth]{./Chapter6_Figures/avgDelayBound_T4_Inj16_DPBPP.eps}
    \rule{35em}{0.5pt}
  \caption[Resulting average delay and average power is shown with a 2-y plot, with T =4+2 symbols, under non-uniform DPBPP traffic under an injection rate of 16 packets/symbol.]{Resulting average delay and average power is shown with a 2-y plot, with T =4+2 symbols, under non-uniform DPBPP traffic under an injection rate of 16 packets/symbol.} 
  \label{fig:Electron}
\end{figure}

First, we evaluate results for T=4+2 symbols of frame length. Fig. 7.21, Fig. 7.22 and Fig. 7.23 show the resulting actual average latency of the packets and resulting average power in terms of minimum transmission power required for 1 RB with BPSK encoding, with a 2-y plot, for an injection rate of 8, 12 and 16 packets/symbol respectively. As you can see from the figures, a diagonal line cuts the plot, which is placed in order to show if the resulting average delay is below the desired average delay bound. 


Firstly, from the figures, we see that for all injection rates, one cannot guarantee an average delay bound lower than 4 symbols due to limited rate and nature of the traffic. Even the algorithm tries its best (with largest $\kappa$ values) it can only provide the lowest possible average delay of approximately 3 symbols, with the highest transmission power, which is convenient. As we increase the average delay bound, we see that we are able to well provide the desired average latency just below the input bound. This shows the strong efficiency of the proposed algorithm. As we expected, increasing the average delay bound makes great power gains. Due to the exponential relation between power and rate, the Pareto like delay-power trade-off is evident in these cases as can be seen from the figures. Even though increasing the average delay provides substantial amount of energy, increasing further and further does not provide the same scale of return as mentioned previously. For instance, for injection rate of 16 packets/symbol, increasing average delay bound from 4 symbols to just 6 symbols, decreases the average power up to 4 times. As expected, with increasing injection rate required average power to provide the same average delay bound gets larger.


\textbf{\textit{Frame Length = 8+2 symbols}}
\begin{figure}[t!]
  \centering
    \includegraphics[width=0.75\textwidth]{./Chapter6_Figures/avgDelayBound_T8_Inj8_DPBPP.eps}
    \rule{35em}{0.5pt}
  \caption[Resulting average delay and average power is shown with a 2-y plot, with T =8+2 symbols, under non-uniform DPBPP traffic under an injection rate of 8 packets/symbol.]{Resulting average delay and average power is shown with a 2-y plot, with T =8+2 symbols, under non-uniform DPBPP traffic under an injection rate of 8 packets/symbol.} 
  \label{fig:Electron}
\end{figure}

\begin{figure}[t!]
  \centering
    \includegraphics[width=0.75\textwidth]{./Chapter6_Figures/avgDelayBound_T8_Inj12_DPBPP.eps}
    \rule{35em}{0.5pt}
  \caption[Resulting average delay and average power is shown with a 2-y plot, with T =8+2 symbols, under non-uniform DPBPP traffic under an injection rate of 12 packets/symbol.]{Resulting average delay and average power is shown with a 2-y plot, with T =8+2 symbols, under non-uniform DPBPP traffic under an injection rate of 12 packets/symbol.} 
  \label{fig:Electron}
\end{figure}

\begin{figure}[t!]
  \centering
    \includegraphics[width=0.75\textwidth]{./Chapter6_Figures/avgDelayBound_T8_Inj16_DPBPP.eps}
    \rule{35em}{0.5pt}
  \caption[Resulting average delay and average power is shown with a 2-y plot, with T =8+2 symbols, under non-uniform DPBPP traffic under an injection rate of 16 packets/symbol.]{Resulting average delay and average power is shown with a 2-y plot, with T =8+2 symbols, under non-uniform DPBPP traffic under an injection rate of 16 packets/symbol.} 
  \label{fig:Electron}
\end{figure}

Next, we continue our evaluation with longer frame lengths, in order to provide different options for computational power for WiNoCoD. Fig. 7.24, Fig. 7.25 and Fig. 7.26 show the same performance graphs for T=8+2 symbols, under injection rates of 8, 12 and 16 packets/symbol respectively.   


We see similar performance for T=8+2 symbols, but with a slightly degraded performance due to more outdated QSI and more infrequent rate re-allocation. Now, average delay bounds lower than 7 symbols cannot be guaranteed. Similarly, after this point increasing average delay bound a few symbols can provide dramatically lower energy expenditure. Also as expected, with longer frame length, to attain the same average delay bound, we have to spend more power for the same injection rate. 





\begin{figure}[b!]
  \centering
    \includegraphics[width=0.75\textwidth]{./Chapter6_Figures/avgDelayBound_T16_Inj8_DPBPP.eps}
    \rule{35em}{0.5pt}
  \caption[Resulting average delay and average power is shown with a 2-y plot, with T =16+2 symbols, under non-uniform DPBPP traffic under an injection rate of 8 packets/symbol.]{Resulting average delay and average power is shown with a 2-y plot, with T =16+2 symbols, under non-uniform DPBPP traffic under an injection rate of 8 packets/symbol.} 
  \label{fig:Electron}
\end{figure}

\begin{figure}[H]
  \centering
    \includegraphics[width=0.75\textwidth]{./Chapter6_Figures/avgDelayBound_T16_Inj12_DPBPP.eps}
    \rule{35em}{0.5pt}
  \caption[Resulting average delay and average power is shown with a 2-y plot, with T =16+2 symbols, under non-uniform DPBPP traffic under an injection rate of 8 packets/symbol.]{Resulting average delay and average power is shown with a 2-y plot, with T =16+2 symbols, under non-uniform DPBPP traffic under an injection rate of 12 packets/symbol.} 
  \label{fig:Electron}
\end{figure}

\begin{figure}[H]
  \centering
    \includegraphics[width=0.75\textwidth]{./Chapter6_Figures/avgDelayBound_T16_Inj16_DPBPP.eps}
    \rule{35em}{0.5pt}
  \caption[Resulting average delay and average power is shown with a 2-y plot, with T =16+2 symbols, under non-uniform DPBPP traffic under an injection rate of 16 packets/symbol.]{Resulting average delay and average power is shown with a 2-y plot, with T =16+2 symbols, under non-uniform DPBPP traffic under an injection rate of 16 packets/symbol.} 
  \label{fig:Electron}
\end{figure} 

\textbf{\textit{Frame Length = 16+2 symbols}}

And finally, the simulations are executed for T=16+2 symbols. As frame lengths get even longer, performance continues to degrade as it can be seen from Fig. 7.27, Fig. 7.28 and Fig. 7.29. 


\section{Information Theoretic Study of the WiNoCoD RF Interconnect}

Previously in this chapter, we have proposed intelligent modulation order selection policies, which increases modulation orders only when necessary, to keep delays of packets bounded. The primary motivation behind this was the exponentially increasing energy expenditure with the modulation orders. We have calculated the required minimum transmission powers according to Shannon's capacity formula \cite{shannon1949mathematical}. As mentioned previously, capacity defines the maximum trasmission rate on a channel, guaranteeing a probability of error approaching to 0, with any protective, reconstructive channel coding. 

We have experimented our proposed algorithms by comparing the resulting average power to the case where the lowest modulation order (BPSK) would be utilized constantly. This has provided us an insight on how much energy can be saved while satisfying certain service demands. 

In this section, we intend to determine minimum required transmission powers (in terms of dBm) for defined information theoretic capacities. For instance, defining transmission powers for spectral capacity densities as 1 bits/sec/Hz, 2 bits/sec/Hz, etc. may provide a good perspective for the respective modulation orders BPSK, QPSK, etc.. For that we make the hypothesis that the transceiver access via transistors to the microstrip transmission line provides a relatively non-varying (non-frequency selective) attenuation over the 20-40 GHz spectrum, in Section 3.3.2.3. If this hypothesis is strong for the via transistor access proposed by M. Hamieh, Fig. 3.12 shows that it is valuable for capacitive coupling but with more attenuation.

In this section, firstly we examine the topology of a U-shaped transmission line, which was previously proposed in the first phase of the project and then we evaluate the cross-shaped transmission line, which was chosen finally due to its much better performance. We will try to dimension the required powers for different communication configurations between tilesets for having spectral capacity densities corresponding to modulation orders. These configurations are unicast communication occuring only between two arbitrary tilesets and the broadcast communication, where a tileset's transmission shall achieve the necessary capacity density even to the farthest node (Any destination tileset should receive packets sent from this tileset with a minimum defined capacity density). In addition to these two configurations, required power for a tileset to achieve an overall capacity density (overall of capacity capacity densities to all other 31 tilesets). 

M. Hamieh has determined that attenuation in the transmission line varies between -0.20 and -0.30 dB/mm over 20-40 GHz spectrum, thus without loss of generality we have chosen to set an average attenuation of -0.25 dB/mm. Wired RF provides not only the advantage of CMOS compatibility, but also significantly less power attenuation compared to wireless interconnects. 

The adjacent transceivers have 8 mm spacing (thus, the signal power attenuates 2 dB between two neighboring tilesets.) And we have assumed a 1 mm of distance for facing tilesets (such as tilesets 1-9, 2-10 etc. in Fig. 7.30). 
Before this, we revisit the well known information theoretic capacity formula of Shannon, defining the highest achievable transmission rate on a channel ensuring a lossless communication, $C_{0}$ in bits/s: 

\begin{equation}
C_{0} = Blog_{2}(1+SNR) 
\end{equation}

where, $B$ is bandwidth in Hz and SNR is Signal-to-Noise Power Ratio in linear. The power of the noise depends on the ambient temperature and bandwidth. For our calculations, we will assume the equally shared scheme where each tileset is allocated 32 subcarriers, thus 640 MHz, uniformly. In other words SNR is the ratio of received signal power $P_{R}$ to the noise power $\frac{P_{R}}{P_{N}}$. It's assumed that noise power $P_{N}$ has a spectral density of -174 dBm/Hz at room temperature \cite{shankar2002introduction}, which we also accept for our calculations. -174 dBm/Hz is equal approximately to $4\,10^{-21}$ W/Hz in linear scale. Therefore, we can rewrite the capacity formula as : 

\begin{equation}
C_{0} = Blog_{2}(1+\frac{P_{R}}{B\,4\,10^{-21}}) 
\end{equation}

Assuming the only attenuation of transmitted signal is due to the losses with distance through the transmission line, which is -0.25 dB/mm, we can rewrite the capacity formula as a function of transmission power :


\begin{equation}
C_{0} = Blog_{2}(1+\frac{P_{T}}{10^{0.025d}\,B\,4\,10^{-21}}) 
\end{equation}

where, $d$ is distance in mm.

\textit{Capacity density} in bits/sec/Hz, $C = C_{0}/B$, is also used, which defines the achievable transmission rate per bandwidth. 

By inverting these formulas, one can also find the required minimum transmission power for our tilesets for a desired capacity density : 

\begin{equation}
P_{T} = 10^{0.025d}\,B\,4\,10^{-21}\,(2^{C}-1) 
\end{equation}

\textit{Required Minimum Transmission Power for desired Bit Error Rate}


As mentioned above, information theoretic capacity is a tool to dimension the commmunication on a channel, defining the upper bound of transmission rate, where probability of error approaches to 0. This is an analytical bound and does not imply any method how to achieve this rate. For instance, theoretically a channel correcting code can achive this rate, where probability of error at the end approaches to 0. As the channel frequency response is flat, we can make an AWGN hypothesis whatever the subcarrier is in 20-40 GHz band. Under this AWGN condition, we can derive the probability of bit error, or with a more common reference, Bit Error Rate (BER) with given transmission power, noise power and transmission rate on a \textit{uncoded} channel. However, it is important to highlight this point : if we use the minimum transmission power for a desired capacity on these BER formulas, we would not get a probability of error of 0. This is because, capacity defines the maximum rate for a coded channel. However, defining required minimum transmission powers for various BER values is still important. 

For BPSK and QPSK, the BER or probability of bit error ($p_{b}$) can be written as \cite{goldsmith2005wireless}: 

\begin{equation}
p_{b} = Q(\sqrt{2\,SNR_{b}})
\end{equation}

where $Q(.)$ is the Gaussian tail function and $SNR_{b}$ is the received SNR per bit. For instance if we are using QPSK (2 bits per constellation symbol) on a bandwidth of 640 MHz, $SNR_{b}$ is calculated by dividing SNR value to 1320 Mbits/s.

For square M-QAM constellation symbols (such as 16-QAM, 64-QAM ..), $p_{b}$ can be written approximately as \cite{goldsmith2005wireless}: 

\begin{equation}
p_{b} = Q\bigg(\sqrt{\frac{3\,SNR_{b}\,b}{2^{b}-1}}\bigg)
\end{equation}

where $b$ is the number of bits per constellation such as 4 bits for 16-QAM, 8 bits for 256-QAM etc. 

Let us calculate the SNR per bit at first. As we did in capacity formula in (7.12), we can write the received SNR as the ratio of transmission power to ambient noise power and attenuation by distance. For BPSK and QPSK : 

\begin{equation}
p_{b} = Q\bigg(\sqrt{\frac{2\,P_{T}}{10^{0.025d}\,B\,4\,10^{-21}}}\bigg)
\end{equation}

and for M-QAM :

\begin{equation}
p_{b} = \frac{4}{b}Q\bigg(\sqrt{\frac{3\,b\,P_{T}}{10^{0.025d}\,B\,4\,10^{-21}\,(2^{b}-1)}}\bigg)
\end{equation}

By inverting these equations we can calculate the required minimum transmission powers with a given BER. The minimum required transmission power for BPSK and QPSK :

\begin{equation}
P_{T} = 0.5\,10^{0.025d}\,B\,4\,10^{-21}(Q^{-1}(p_{b})^{2})
\end{equation}

and minimum required transmission power for M-QAM constellations :

\begin{equation}
P_{T} = \frac{1}{3}\,10^{0.025d}\,B\,4\,10^{-21}(Q^{-1}(0.25\,p_{b}\,b))^{2}\,(2^{b}-1)
\end{equation}


\subsection{U-Shaped Transmission Line}

The first topology proposed was a U-shaped transmission line as in Fig. 7.30. Not having a circular loop (close ends) avoids interfering reflections of the tranmitted signal.  


\begin{figure}[htbp]
  \centering
    \includegraphics[width=0.8\textwidth]{./Figures/transline_U.pdf}
    \rule{35em}{0.5pt}
  \caption[U-shaped transmission line]{U-shaped transmission line}
  \label{fig:Electron}
\end{figure}

 

\subsubsection{Unicast communication}
Fig. 7.31 shows the distances from any tileset to any other tileset in U-shaped transmission line (32x31 combinations of unicast communications). As it can be seen from Fig. 7.31 that the largest distance (between tilesets at the opposite ends of the transmission line) is 120 mm, which causes an attenuation of -40 dB. Even though our OFDMA based RF interconnect is intended to provide a fully broadcast capable communication infrastructure, firstly we have analyzed the information theoretic limits for each of the 32x31 unicast communication combination between tilesets.

\begin{figure}[htbp]
  \centering
    \includegraphics[width=0.99\textwidth]{./Figures/linearTransLine_Distance.eps}
    \rule{35em}{0.5pt}
  \caption[Distance between each 32x31 unicast communication in U-shaped transmission line]{Distance between each 32x31 unicast communication in U-shaped transmission line}
  \label{fig:Electron}
\end{figure}

For these unicast communication combinations we can write the capacities in a matrix form for source-destination tileset pairs, $C_{0}^{ij}$, where $N$ is the number of tilesets : 

\bigskip
\resizebox{1.0\linewidth}{!}
{
$\begin{bmatrix}
 0  &Blog_{2}(1+\frac{P_{T}}{10^{0.025d_{12}}\,B\,4\,10^{-21}})  &Blog_{2}(1+\frac{P_{T}}{10^{0.025d_{13}}\,B\,4\,10^{-21}})  &.  &. &Blog_{2}(1+\frac{P_{T}}{10^{0.025d_{1N}}\,B\,4\,10^{-21}}) \\ 
Blog_{2}(1+\frac{P_{T}}{10^{0.025d_{21}}\,B\,4\,10^{-21}}) &0   &Blog_{2}(1+\frac{P_{T}}{10^{0.025d_{23}}\,B\,4\,10^{-21}})  &.  &. &Blog_{2}(1+\frac{P_{T}}{10^{0.025d_{2N}}\,B\,4\,10^{-21}}) \\ 
Blog_{2}(1+\frac{P_{T}}{10^{0.025d_{12}}\,B\,4\,10^{-31}})   &Blog_{2}(1+\frac{P_{T}}{10^{0.025d_{32}}\,B\,4\,10^{-21}})  &0 &.  &. &Blog_{2}(1+\frac{P_{T}}{10^{0.025d_{3N}}\,B\,4\,10^{-21}}) \\ 
 .  &. &.  &. &.  &. \\ 
 .  &. &.  &. &.  &. \\ 
  &Blog_{2}(1+\frac{P_{T}}{10^{0.025d_{N1}}\,B\,4\,10^{-21}})  &Blog_{2}(1+\frac{P_{T}}{10^{0.025d_{N2}}\,B\,4\,10^{-21}}) &. &.  &0  
\end{bmatrix}$
}
\bigskip

%\centerline{(7.20)}

Assuming a transmission power of -80 dBm per tileset the received capacities for unicast communication combinations are shown in Fig. 7.32. Note that, with this transmission power while neighboring nodes can achieve capacities up to 3 Gbps on their 640 MHz bandwidth, the farthest nodes' capacity approaches to 0. From an information theoretic perspective, this means that with this transmission power these nodes cannot have a reliable communication.

\begin{figure}[t!]
  \centering
    \includegraphics[width=0.99\textwidth]{./Figures/linearTransLine_capacity01.eps}
  \caption[Capacity between each 32x31 unicast communication in U-shaped transmission line]{Capacity between each 32x31 unicast communication in U-shaped transmission line}
  \label{fig:Electron}
\end{figure}

Next, based on this distances, we calculate the required transmission power for each of 32x31 unicast communication combination between tilesets. We assume that, each tileset uses 32 subcarriers (thus, 640 MHz) for transmission and the noise spectral density at standard room temperature is assumed (-174 dBm/Hz). Note that, the required transmission power can be calculated simply by adjusting these results linearly with increasing or decreasing bandwidth (number of subcarriers). The required transmission powers for unicast source-destination tileset pairs as a function of distance and channel capacity spectral density is as follows :

\bigskip
\resizebox{1.0\linewidth}{!}
{
$\begin{bmatrix}
 0  &P_{T} = 10^{0.025d_{12}}\,B\,4\,10^{-21}\,(2^{C}-1)  &P_{T} = 10^{0.025d_{13}}\,B\,4\,10^{-21}\,(2^{C}-1)  &.  &. &P_{T} = 10^{0.025d_{1N}}\,B\,4\,10^{-21}\,(2^{C}-1) \\ 
P_{T} = 10^{0.025d_{21}}\,B\,4\,10^{-21}\,(2^{C}-1) &0   &P_{T} = 10^{0.025d_{23}}\,B\,4\,10^{-21}\,(2^{C}-1)  &.  &. &P_{T} = 10^{0.025d_{2N}}\,B\,4\,10^{-21}\,(2^{C}-1) \\ 
P_{T} = 10^{0.025d_{31}}\,B\,4\,10^{-21}\,(2^{C}-1)   &P_{T} = 10^{0.025d_{32}}\,B\,4\,10^{-21}\,(2^{C}-1)  &0 &.  &. &P_{T} = 10^{0.025d_{3N}}\,B\,4\,10^{-21}\,(2^{C}-1) \\ 
 .  &. &.  &. &.  &. \\ 
 .  &. &.  &. &.  &. \\ 
  &P_{T} = 10^{0.025d_{N1}}\,B\,4\,10^{-21}\,(2^{C}-1)  &P_{T} = 10^{0.025d_{N2}}\,B\,4\,10^{-21}\,(2^{C}-1) &. &.  &0  
\end{bmatrix}$
}
\bigskip


%\centerline{(7.21)}


Based on the attenuation values due to distances in Fig. 7.31, the required transmission power for each of these unicast communication combinations in dBm are calculated and shown in Fig. 7.33, for different spectral channel densities between 1 bits/sec/Hz and 8 bits/sec/Hz, corresponding to different modulation orders. 

\begin{figure}[H]
  \centering
    \includegraphics[width=0.99\textwidth]{./Figures/linearTransLine_MinPower_CapDen_1_8bits.eps}
    \rule{35em}{0.5pt}
  \caption[Required Transmission Power for each 32x31 unicast communication for the U-shaped shaped transmission line for capacity densities 1-8 bits/s/Hz]{Required Transmission Power for each 32x31 unicast communication for the U-shaped shaped transmission line for capacity densities 1-8 bits/s/Hz}
  \label{fig:Electron}
\end{figure}

We have determined the required transmission powers for attaining the channel capacity densities for each possible 32x31 unicast communication. However, assuming that a tileset may communicate any other tileset throughout the execution of the application (not only one), providing an average value of the previously determined minimum transmission power values is utile. In other words, for each of 32 tilesets, we shall give an average of the unicast required transmission power values over each 31 distance tilesets as :

\begin{equation}
P_{T}^{avg}(i) = \frac{1}{31} \sum_{i \neq j, 1<i<32}^{}    P_{T}(d(i,j))
\end{equation}

Fig. 7.34 shows the average required transmission power for each tileset over its 31 destination tilesets, for the U-shaped topology, for capacity densities 1-8 bits/s/Hz. Note that, as tilesets position gets nearer to the center of transmission line (i.e. near tileset-16), required trasnmission power decreases and as its position gets nearer to the edges of transmission line (i.e. near tileset-1 or tileset-32), required transmission power increases. This is due to the fact that average distance to tilesets gets higher as we approach to the edges, and gets lower as we approach to the center. And as attenuation in dB increases linearly with distance, this significantly affects the transmission power. From this figure, we can understand that if a centralized mechanism is used, it is much more efficient to place the CIU inside the RF transceiver of Tileset-15, Tileset-16, Tileset-17 or Tileset-18, as they require much less transmission power due to their distances to other tilesets. 
    
    
\begin{figure}[H]
  \centering
    \includegraphics[width=0.99\textwidth]{./Figures/linearTransLine_overall_minPow_dBm_18cap.eps}
    \rule{35em}{0.5pt}
  \caption[Average required transmission power (to 31 destinations) for each tileset for the U-shaped shaped transmission line for capacity densities 1-8 bits/s/Hz]{Average required transmission power (to 31 destinations) for each tileset for the U-shaped shaped transmission line for capacity densities 1-8 bits/s/Hz}
  \label{fig:Electron}
\end{figure}

Following this, we provide the average minimum transmission power of each tileset (average to 31 destinations), for BERs of $10^{-1}, 10^{-3}, 10^{-5}, 10^{-7}$ for BPSK and 256-QAM (the lowest and highest modulation orders). In order to have a reference, we also include the average required minimum transmission powers for information theoretic channel capacity densities of 1 bits/s/Hz and 8 bits/s/Hz, corresponding to BPSK and 256-QAM. Again, we remind that information theoretic capacity defines the maximum rate on a coded channel where probability of error approaches to 0 (whereas, it does not define this hypothetical error correcting code explicitly). However, the average power values given here for different probabilities of error are for uncoded channels. Hence, it is possible that we might need larger minimum transmission powers for certain probabilities of error larger than 0, compared to channel capacities (where probability of error approaches to 0). 

    
\begin{figure}[H]
  \centering
    \includegraphics[width=0.99\textwidth]{./Chapter6_Figures/linearTransLine_overall_minPow_dBm_errorRate.eps}
    \rule{35em}{0.5pt}
  \caption[Average required transmission power (to 31 destinations) for each tileset for the U-shaped shaped transmission line for probabilities of error : $10^{-1}$, $10^{-3}$, $10^{-5}$, $10^{-7}$ for BPSK and 256-QAM]{Average required transmission power (to 31 destinations) for each tileset for the U-shaped shaped transmission line for probabilities of error : $10^{-1}$, $10^{-3}$, $10^{-5}$, $10^{-7}$ for BPSK and 256-QAM}
  \label{fig:Electron}
\end{figure}

Observing Fig. 7.35, we see that for BPSK, for a probability of error of $10^{-1}$, tilesets require a little bit less average transmission power compared to power required for information theoretic channel capacity of 1 bits/s/Hz. For probabilities of error of $10^{-3}$, $10^{-5}$ and $10^{-7}$, we see that we need higher transmission powers. Even though these probability of errors are larger than 0, they require larger powers as these formulas are for uncoded transmission. Of course, in order to obtain results in accordance of intra-chip communication requirements, channel coding should be added. Hence, for a BER of $10^{-7}$ before decoding, an order of $10^{-14}$ can be achieved after usual realistic coding; but this is out of scope of this study.


\subsubsection{Broadcast communication}

As we have reviewed in this thesis previously, tilesets broadcast their packets and the other tilesets check the associated flags in header flits to understand whether this packet is for it or not. Therefore, we must ensure a reliable communication for packets concerning every other possible destination. In addition to this, we have mentioned that the on-chip traffic possess high amount of broadcast packets. Hence, determining required minimum transmission power by concerning a capacity density even to the farthest destination tileset is essential. Fig. 7.36 shows the required transmission powers for each tileset to its farthest destination, for the U-shaped topology, for capacity densities 1-8 bits/s/Hz. As we have stated in previous section, as tilesets position gets more to the center of the transmission line, its distance to the farthest destination node also decreases, which is decreasing the required transmission power significantly. 

\begin{figure}[H]
  \centering
    \includegraphics[width=0.99\textwidth]{./Figures/linearTransLine_broadcast_minPow_dBm_18cap.eps}
    \rule{35em}{0.5pt}
  \caption[Minimum required transmission power for broadcasting (to maximum distance) for each tileset for the U-shaped shaped transmission line for capacity densities 1-8 bits/s/Hz]{Minimum required transmission power for broadcasting (to maximum distance) for each tileset for the U-shaped shaped transmission line for capacity densities 1-8 bits/s/Hz}
  \label{fig:Electron}
\end{figure}

Fig. 7.37 shows the required minimum transmission powers for tilesets for broadcasting (to farthest destination node), for probabilities of error of $10^{-3}$, $10^{-5}$ and $10^{-7}$, under BPSK and 256-QAM, compared to powers required for correspending channel capacities 1 bits/s/Hz and 8 bits/s/Hz. We see a similar pattern as we observed for average unicast transmission power values. 


    
\begin{figure}[H]
  \centering
    \includegraphics[width=0.99\textwidth]{./Chapter6_Figures/linearTransLine_broadcast_minPow_dBm_errorRate.eps}
    \rule{35em}{0.5pt}
  \caption[Required transmission power for broadcasting (farthest destination) for each tileset for the U-shaped shaped transmission line for probabilities of error : $10^{-1}$, $10^{-3}$, $10^{-5}$, $10^{-7}$ for BPSK and 256-QAM]{Required transmission power for broadcasting (farthest destination) for each tileset for the U-shaped shaped transmission line for probabilities of error : $10^{-1}$, $10^{-3}$, $10^{-5}$, $10^{-7}$ for BPSK and 256-QAM}
  \label{fig:Electron}
\end{figure}



\begin{table}[b!]
\centering
\caption{Total of maximum transmission powers (broadcast case) for various capacity densities and BER values with BPSK or 256-QAM with cross-shaped transmission line.}
\label{my-label}
\begin{tabular}{|c|c|}
\hline
U-Shaped                               & \begin{tabular}[c]{@{}c@{}} Total of Maximum \\ Transmission Powers \\ (Broadcast case) - dBm\end{tabular} \\ \hline
Capacity 1 bits/s/Hz                       & -30.75  \\ \hline
Capacity 2 bits/s/Hz                       & -25.97  \\ \hline
Capacity 3 bits/s/Hz                       & -22.29 \\ \hline
Capacity 4 bits/s/Hz                       & -18.98 \\ \hline
Capacity 5 bits/s/Hz                       & -15.83 \\ \hline
Capacity 6 bits/s/Hz                       & -12.75                                                                                                    \\ \hline
Capacity 7 bits/s/Hz                       & -9.71 \\ \hline
Capacity 8 bits/s/Hz                       & -6.68 \\ \hline
BPSK $p_{b}=10^{-1}$     & -31.6                                                                                                    \\ \hline
BPSK $p_{b}=10^{-1}$     & -23.96 \\ \hline
BPSK $p_{b}=10^{-1}$     & -21.16                                                                                                    \\ \hline
BPSK $p_{b}=10^{-1}$   & -19.44 \\ \hline
256-QAM $p_{b}=10^{-1}$   & -12.95                                                                                                    \\ \hline
256-QAM $p_{b}=10^{-1}$  & -2.27                                                                                                    \\ \hline
256-QAM $p_{b}=10^{-1}$  & 0.81 \\ \hline
256-QAM $p_{b}=10^{-1}$  & 2.64 \\ \hline
\end{tabular}
\end{table}

Table 7.3 gathers the total values (total of all 32 tilesets) for maximum transmission powers (i.e. required power for transmission to farthest node in broadcast case) for capacity densities between 1-8 bits/s/Hz and various BERs with BPSK or 256-QAM with a U-shaped transmission line. In other words, this table represents the total of values in Fig. 7.36 and Fig. 7.37. From this table, we can have an idea on the worst case, where highest transmission power budget is required.  For instance, total required power for providing broadcasting with our CMP, for a capacity density of 8 bits/s/Hz is -6.68 dBm.

\subsection{Cross-Shaped Transmission Line}

We have mentioned previously in Section 3.3.2.3 that a cross-shaped transmission line topology was chosen later on the simple U-shaped topology. This cross-shaped topology decreases distances among tilesets and also diminishes the frequency selective channel attenuation. The chosen cross-shaped topolgy with positioned tileset-IDs is illustrated in Fig. 7.38. 

 
\begin{figure}[H]
  \centering
    \includegraphics[width=0.8\textwidth]{./Figures/cross_transline.pdf}
    \rule{35em}{0.5pt}
  \caption[Cross-shaped transmission line]{Cross-shaped transmission line}
  \label{fig:Electron}
\end{figure}

Next, we determine the required transmission powers for attaining desired capacity densities for our cross-shaped transmission line and compare them to the previous U-shaped transmission line. 

\subsubsection{Unicast communication}

Finding distances among tilesets is not trivial as for the U-shaped transmission line due to irregularity of the topology. Firstly, we define a \textit{symmetry} among tilesets, that who's distances to the 31 other tilesets would be same. For instance, from Fig. 7.35, we see that tileset-4 and tileset-29 have a positional symmetry. Tileset-4's distance to tileset-5 is same with tileset-29's distance to tileset-28 etc. Fig. 7.39 shows the distances in mm among 32x31 combinations of unicast communication on cross-shaped transmission line. Note that, the maximum distance in cross-shaped transmission line is 80 mm, while maximum distance in U-shaped transmission line is 120 mm. With an average attenaution of -0.25 dB/mm, this means that the largest attenaution is 10 dB more (10 times in scalar) for the U-shaped transmission line. 

\begin{figure}[H]
  \centering
    \includegraphics[width=0.99\textwidth]{./Figures/crossTransLine_Distance.eps}
    \rule{35em}{0.5pt}
  \caption[Distance between each 32x31 unicast communication in cross-shaped transmission line]{Distance between each 32x31 unicast communication in cross-shaped transmission line}
  \label{fig:Electron}
\end{figure}

Similarly we did for the U-shaped transmission line, we present the achievable capacities for transmitting-recepting tileset pairs with -80 dBm of transmission for power on the cross-shaped transmission line. By comparing Fig. 7.40 to 7.32, we see that on both of the topologies neighboring tilesets achieve a capacity of 3 Gbps as expected and farthest nodes' capacity still approaches to 0, however the achieved capacities for distances between these two extremum points are remarkably improved.

\begin{figure}[H]
  \centering
    \includegraphics[width=0.99\textwidth]{./Figures/crossTransLine_capacity01.eps}
  \caption[Capacity between each 32x31 unicast communication in cross-shaped transmission line]{Capacity between each 32x31 unicast communication in cross-shaped transmission line}
  \label{fig:Electron}
\end{figure}

Fig. 7.41 shows the required transmission power in dBm for 32x31 unicast communication combination on the cross-shaped transmission line for capacity densities between 1-8 bits/s/Hz. We observe that required transmission powers are lowered by comparing it to U-shaped transmission line. 

\begin{figure}[H]
  \centering
    \includegraphics[width=0.99\textwidth]{./Figures/crossTransLine_MinPower_CapDen_1_8bits.eps}
    \rule{35em}{0.5pt}
  \caption[Required Transmission Power for each 32x31 unicast communication for the cross-shaped shaped transmission line for capacity densities 1-8 bits/s/Hz]{Required Transmission Power for each 32x31 unicast communication for the cross-shaped shaped transmission line for capacity densities 1-8 bits/s/Hz}
  \label{fig:Electron}
\end{figure}


Fig. 7.42 shows the average required transmission power for each tileset over its 31 destination tilesets, for the U-shaped topology, for capacity densities 1-8 bits/s/Hz. One can see the reduction of the energy expenditure thanks to lower distances. From this figure, we can understand that if a centralized mechanism is used, it is much more efficient to place the CIU inside the RF transceiver of Tileset-3, Tileset-10, Tileset-20 or Tileset-29, as they require much less transmission power due to their distances to other tilesets. 
 
\begin{figure}[H]
  \centering
    \includegraphics[width=0.99\textwidth]{./Figures/crossTransLine_overall_minPow_dBm_18cap.eps}
    \rule{35em}{0.5pt}
  \caption[Average required transmission power (to 31 destinations) for each tileset for the cross-shaped shaped transmission line for capacity densities 1-8 bits/s/Hz]{Average required transmission power (to 31 destinations) for each tileset for the cross-shaped shaped transmission line for capacity densities 1-8 bits/s/Hz}
  \label{fig:Electron}
\end{figure}


As we did for the U-shaped transmission line, we present the average minimum transmission power for each tileset (average of 31 destinations) for BPSK and 256-QAM, for probability of bit errors $10^{-1}$, $10^{-3}$, $10^{-5}$, $10^{-7}$. We observe similar patterns in Fig 7.43.
    
\begin{figure}[H]
  \centering
    \includegraphics[width=0.99\textwidth]{./Chapter6_Figures/crossTransLine_overall_minPow_dBm_errorRate.eps}
    \rule{35em}{0.5pt}
  \caption[Average required transmission power (to 31 destinations) for each tileset for the cross-shaped shaped transmission line for probabilities of error : $10^{-1}$, $10^{-3}$, $10^{-5}$, $10^{-7}$ for BPSK and 256-QAM]{Average required transmission power (to 31 destinations) for each tileset for the cross-shaped shaped transmission line for probabilities of error : $10^{-1}$, $10^{-3}$, $10^{-5}$, $10^{-7}$ for BPSK and 256-QAM}
  \label{fig:Electron}
\end{figure}



\subsubsection{Broadcast communication}

Fig. 7.44 shows the required transmission powers for each tileset to its farthest destination, for the cross-shaped topology, for capacity densities 1-8 bits/s/Hz. We see the improvements due to reduction of maximum distances. For instance, compared to U-shaped transmission line, required transmission powers for broadcast communications are reduced between 15 and 16 dBm.  

\begin{figure}[htbp]
  \centering
    \includegraphics[width=0.99\textwidth]{./Figures/crossTransLine_broadcast_minPow_dBm.eps}
    \rule{35em}{0.5pt}
  \caption[Minimum required transmission power for broadcasting (to maximum distance) for each tileset for the cross-shaped shaped transmission line for capacity densities 1-8 bits/s/Hz]{Minimum required transmission power for broadcasting (to maximum distance) for each tileset for the cross-shaped shaped transmission line for capacity densities 1-8 bits/s/Hz}
  \label{fig:Electron}
\end{figure}


And finally, Fig. 7.45, shows the minimum transmission power for various probabilities of error for BPSK and 256-QAM for broadcasting, concerning the farthest destination for each tileset on the cross-shaped transmission line. 

\begin{figure}[htbp]
  \centering
    \includegraphics[width=0.99\textwidth]{./Chapter6_Figures/crossTransLine_broadcast_minPow_dBm_errorRate.eps}
    \rule{35em}{0.5pt}
  \caption[Required transmission power for broadcasting (farthest destination) for each tileset for the U-shaped shaped transmission line for probabilities of error : $10^{-1}$, $10^{-3}$, $10^{-5}$, $10^{-7}$ for BPSK and 256-QAM]{Required transmission power for broadcasting (farthest destination) for each tileset for the U-shaped shaped transmission line for probabilities of error : $10^{-1}$, $10^{-3}$, $10^{-5}$, $10^{-7}$ for BPSK and 256-QAM}
  \label{fig:Electron}
\end{figure}



\begin{table}[]
\centering
\caption{Total of maximum transmission powers (broadcast case) for various capacity densities and BER values with BPSK or 256-QAM with cross-shaped transmission line.}
\label{my-label}
\begin{tabular}{|c|c|}
\hline
Cross-Shaped                               & \begin{tabular}[c]{@{}c@{}} Total of Maximum \\ Transmission Powers \\ (Broadcast case) - dBm\end{tabular} \\ \hline
Capacity 1 bits/s/Hz                       & -42.21                                                                                                    \\ \hline
Capacity 2 bits/s/Hz                       & -37.44                                                                                                    \\ \hline
Capacity 3 bits/s/Hz                       & -33.76                                                                                                    \\ \hline
Capacity 4 bits/s/Hz                       & -30.45                                                                                                    \\ \hline
Capacity 5 bits/s/Hz                       & -27.29                                                                                                    \\ \hline
Capacity 6 bits/s/Hz                       & -24.21                                                                                                    \\ \hline
Capacity 7 bits/s/Hz                       & -21.17                                                                                                    \\ \hline
Capacity 8 bits/s/Hz                       & -18.14                                                                                                    \\ \hline
BPSK $p_{b}=10^{-1}$     & -43.06                                                                                                    \\ \hline
BPSK $p_{b}=10^{-1}$     & -35.42                                                                                                    \\ \hline
BPSK $p_{b}=10^{-1}$     & -32.62                                                                                                    \\ \hline
BPSK $p_{b}=10^{-1}$   & -30.9                                                                                                     \\ \hline
256-QAM $p_{b}=10^{-1}$  & -24.41                                                                                                    \\ \hline
256-QAM $p_{b}=10^{-1}$  & -13.73                                                                                                    \\ \hline
256-QAM $p_{b}=10^{-1}$  & -10.64                                                                                                    \\ \hline
256-QAM $p_{b}=10^{-1}$  & -8.81                                                                                                     \\ \hline
\end{tabular}
\end{table}

Table 7.4 gathers the total values (total of all 32 tilesets) for maximum transmission powers (i.e. required power for transmission to farthest node in broadcast case) for capacity densities between 1-8 bits/s/Hz and various BERs with BPSK or 256-QAM with a cross-shaped transmission line. In other words, this table represents the total of values in Fig. 7.44 and Fig. 7.45. For instance, total required power for providing broadcasting with our CMP, for a capacity density of 8 bits/s/Hz is -18.14 dBm, which is approximately 12 dB lower compared to U-shaped topology.
\section{Conclusion}

OFDMA not only provides the oppurtunity of digitally and rapidly re-arbitrate the frequency resources, i.e. \textit{subcarriers}, but also the chance of selecting different modulation orders, therefore different \textit{rates} as a new flexible dimension of scalable rates for future on-chip interconnects with a trivial digital procedure.

Due to computational/circuitral difficulty and additional bandwidth overhead of signaling between tilesets, we have constructed a framework, where a single modulation order is selected for each tileset every frame. This modulation order selection procedure is completely based on the framed bandwidth allocation framework, explained in previous chapters. In this chapter, we have evaluated the relation between delay and power for the on-chip RF interconnect of WiNoCoD. 


Based on our experiments, we can state firmly that average delay bounded algorithm is robust and provides much more reliable results compared to maximum delay bounded algorithm. We believe, setting bounds on averaged delays may be a more convenient way for the highly dynamic and fluctuating on-chip traffic. We have shown that, average power can be reduced up to 10 times, by increasing average delay bound by a few symbols. As a last remark, we should emphasize that this algorithm bases itself on the equations from the reference paper, except we have added the number of channels to them and modified for the WiNoCoD's framed structure.   

For these proposed intelligent modulation order selection policies, the required transmission power for different modulation orders were quantified as a multiple of the power required for BPSK. This omits the calculations of the real transmission powers that tilesets shall supply. Therefore, we have determined the destinations between tilesets for the previously proposed U-shaped topology and the improved latest cross-shaped topology. Based on these, by using the determined signal attenuation with respect to distance in our state-of-the-art transmission line, we have calculated the required minimum transmission powers concerning the information theoretic bounds. Required transmission powers for different channel capacity densities between 1-8 bits/sec/Hz are determined, which correspond to modulation orders between BPSK to 256-QAM. 


