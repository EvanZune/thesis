% Chapter 8

\chapter{Conclusions and Perspectives} % Main chapter title


\label{Chapter8} % For referencing the chapter elsewhere, use \ref{Chapter1} 

\lhead{Chapter 8. \emph{Conclusions and Perspectives}} % This is for the header on each page - perhaps a shortened title

In this thesis work which has been performed in parallel with Project WiNoCoD, we have demonstrated the feasibility of utilizing OFDMA as a modulation mechanism on an on-chip wired RF interconnect for a massive manycore architecture. By using newly enabled necessary CMOS technology for the development of fast enough components, WiNoCoD Project aims to be the first attempt to bring OFDMA on-chip. 

In Chapter 2, we have explained the emerging Chip Multiprocessor (CMP) paradigm and its basic computational structure. The need for cache coherency was discussed briefly, which is the contributor of the traffic between on-chip elements. However, electrically wired NoCs cannot sustain the traffic demand of the CMPs with hundreds of cores. The current estimation predicts that number of cores on a single CMP may reach several thousands before the end of next decade. Therefore, the state-of-the-art optical and RF interconnects are proposed for CMPs with hundreds and thousands of cores. However, these interconnects are static due to their reliance on analog circuits for generating orthogonal frequency channels. Hence, these interconnects shall be dimensioned according to peak traffic of a node. Further in the chapter, we discuss the statistical properties and burst nature of on-chip traffic and present certain synthetic traffic models to emulate cache coherency traffic.

Chapter 3 presents the Project WiNoCoD and its 2048-core CMP. First, we explain the scalable 3-level NoC hierarchy, along with each layer's specific communication infrastructure. We also mention the utilized scalable distributed hybrid cache coherency protocol (DHCCP) and its intensive demand of broadcast messages. Later in the chapter, before presenting the details of our wired RF interconnect, we briefly explain the basics of OFDM and OFDMA, and its potential advantages for the proposed on-chip architecture. OFDMA is broadcast capable intrinsically and orthogonal channel generation and bandwidth allocation in OFDMA is purely digital, rapid and efficient in contrast with the previously mentioned optical and RF interconnects. Furthermore, circuitry and parameters of our wired RF interconnect are presented in detail.

Chapter 4 firstly formulates the main problem of this thesis from queuing and network theory perspective, which is to find the right algorithms to distribute frequency resources to the multiple transmission queues, such that certain metrics of interest like average latency is minimized. Certain algorithms from the literature approaching this issue from different grounds and their theoretical basis are discussed. Next, the infrastructure of WiNoCoD's OFDMA based RF interconnect and certain preliminary concepts are introduced. For instance, due to signaling and computational constraints, the subcarriers in the system are grouped to serve 1 flit on a single system. The unusually large OFDM bandwidth compared to most of the existing in our system, forces resource allocation algorithms to be done in few hundrerds of nanoseconds. Taking into account other additional factors increasing the cumulative latency for the effectuating bandwidth allocation, such as digital reconfiguration of the components, propagation time, computation time of OFDM components such as IFFT etc., a pipelined mechanism for subcarrier arbitration is adopted. The frame based bandwidth allocation based on this pipelined fashion is explained in detail. Two cardinally different methods for bandwidth allocation are considered in WiNoCoD : Decentralized mechanism, where each tileset broadcasts their QSIs and execute the same algorithm or centralized mechanism where a central intelligent unit (CIU) is responsible for this. The coordination between tilesets and other tilesets or between tilesets and the CIU for signaling is discussed for both of the cases. Pros and cons of two different methods are listed. In addition, utilized methods and realistic traffic models to stress algorithms are explained. Different granularity levels for signaling and resource partition was discussed and best option was chosen to implement bandwidth allocation algorithms.  

In Chapter 5 we present and evaluate our proposed algorithms experimentally. First proposed method is called serial QSI allocation algorithm, where resource blocks (RBs) are sequentially allocated in blocks by iterating through QSI demands of tilesets. A major motivation behind this algorithm is to minimize the computational complexity and number of iterations as much as possible. Even though this algorithm provides very low average latencies and delay or queue exceeding probabilities, we have observed that it has a limited capacity due to its unfair nature. This phenomenon was examined and certain extension to enhance the algorithm with minimum additional complexity are presented. Next proposed algorithm was queue proportional scheduling (QPS), which allocates the available frequency resources proportional to queue lengths of tilesets. The capacity of the system was proven to be efficient, however we have observed that under lower traffic loads, the average latencies were unacceptably high. Using modified QSIs such as deterministic or expected QSI to combat outdated QSI and allow for a fairer resource allocation are evaluated for both of these methods for both centralized and decentralized mechanisms. We conclude that there is no \textit{ultimate best algorithm}, where they offer different good performances for different metrics of interest under different configurations and traffic statistics. However, using serial allocation algorithm with definitive QSI encoding was proven to provide the lowest average latency in general. This basic algorithm which does not require any computation complexity is feasible to be implemented with very low frame lengths such as few OFDM symbols. However, especially under higher traffic loads, we have observed QPS algorithms was able to outperform basic serial allocation in terms of remarkably lower packet delay and queue length exceeding probabilities. 
 
In Chapter 6, one of the most innovative bandwidth allocation infrastructure we have developed for an on-chip interconnect is presented, where there exists no similar attempt in the literature, to the best of our knowledge. This novel Payload Channel Algorithm allocates all the subcarriers on a symbol to the payloads of cache line carrying long cache coherency packets. Thanks to intrinsic broadcast nature of OFDMA, this algorithm does not require any extra signaling overhead, as the type of the packet is encoded in header flits. Additionaly, two separate queues are used to avoid any inconsistency between headers and packets and also to be able to utilize resources while waiting for payload transmission. This algorithm was proven to decrease average latency up to 10 times under certain cases compared to a static counterpart. In addition, we derived an analytical expression for the average latency of this algorithm as a function of injection rate by using certain approximations and queuing theory. Furthermore, a more sophisticated dynamic version of this algorithm is developed, which allocates base resources to tilesets according to traffic fluctuations. We believe this algorithm may provide substantial performance increase for the future CMPs using OFDMA, especially with longer cache lines.


Chapter 7 is dedicated to option of using different modulation orders for WiNoCoD's OFDMA interconnect. Firstly, information theoretic relation between delay and power was revisited. Then, the necessary mechanism for using higher modulation order for extra transmission power in WiNoCoD was explained, briefly. Both inspired from the existing schedulers from the literature but designed for general wireless telecommunications, two novel modulation order selection algorithms are introduced. First one aims to minimize transmission power while setting a maximum delay of the packets. The original algorithm was designed for a single user-single channel case, which was not applicable to our case directly. Therefore, we have modified this scheduler for this special case, which we believe can also be applied to more generic multi-user multi-channel cases in wireless telecommunications. This algorithm is able to provide 3 times lower average power by increasing delay bound by just 16 symbols under certain scenarios. Moreover, our scheduler is not able to provide the necessary delay bound especially under bursty traffic due to limited rate of the system, however it was shown that a probabilistic guarantee can be achieved. Second algorithm was also based on an existing scheduler from the literature, which we have modified to our case. Different than the previous one, this algorithm aims to minimize the average transmission power while setting a bound on average delay. The proposed scheduler was able to decrease the average energy expenditure 4 times by just increasing the average latency by few symbols. Algorithm was shown to be efficient in terms of tracing and obeying the average delay bound. We have also conducted an information theoretic analysis on two different transmission line topologies in this chapter. Proposed algorithms for intelligent modulation order choice seek to select lowest modulation order (BPSK) as possible as long as the service criteria on latency are met and give the resulting power consumption relative to BPSK. We have not given explicit transmission power values previously, however by this information theoretic analysis part we aim to determine the required minimum transmission power to achieve certain capacity and uncoded bit error rate.


\textit{\textbf{Perspectives}}

WiNoCoD Project is an ambitious endeavor to break the reconfigurable bandwidth allocation bottleneck for massive manycore systems. We believe the proposed algorithms and discussed concepts shall be a pioneering guideline for the future OFDMA based on-chip interconnects.

In addition to this work conducted throughout the project, there exists a great potential for further contribution. With its reconfigurability, once OFDMA RF on-chip interconnects have become feasible for production, researchers may develop certain extensive algorithms and infrastructures. For instance, \textit{carrier aggregation} paradigm from cellular communications may be investigated, that deals with the allocation of non-contiguous subcarriers to nodes, which can decrease peak-to-average power ratio problem and provide other certain benefits. Another perspective can be the \textit{offline} optimization of subcarrier and modulation order allocation, by inspecting the traffic demand of nodes, after executing applications on the CMP. 

Further work could try to evaluate the transmission power consumption both for static modulation order or dynamic modulation order case, for various traffic patterns. New energy aware intelligent modulation order selection algorithms can be developed.

By developing silicon technology in future, we can expect to have several hundreds of GHz of bandwidth which results in much shorter OFDM symbol durations. The proposed pipelined framework and bandwidth allocation mechanism can be extended to these extreme cases. In addition, we can expect to have more number of processing elements, thus more elements accessing the RF transmission line. Also higher number of subcarriers can be implemented with future technology to sustain communication for this much number of RF nodes. The work conducted in this thesis can be extended to these cases as well.  

Further work can be done on the physical layer aspects of this OFDMA based interconnect. The Peak-to-Average Power Ratio (PAPR) problem can be investigated and optimization on arbitration of subcarriers and modulation can be performed concerning this problem. In addition, channel coding and other error detection and correction techniques can be developed taking in to account the very specific constraints of this interconnect. Even though we have not taken into account the variation of attenuation between the 20-40 GHz spectrum, and assumed an average flat frequency response; further optimal subcarrier allocation algorithms can be developed, which also try to minimize the attenuation. 

    
