% !Mode:: "TeX:UTF-8"
% Chapter 0

\chapter{Résumé en Français} % Main chapter title

%\label{Chapter0} % For referencing the chapter elsewhere, use \ref{Chapter1} 

\lhead{\emph{Résumé en Français}} % This is for the header on each page - perhaps a shortened title

La densité du trafic des communications sans fil n’a cessé d’augmenter depuis 2 décennies. Les futures technologies de réseaux cellulaires visent à supporter un trafic toujours plus élevé, grâce à des réseaux auto-organisés offrant une meilleure capacité tout en réduisant la consommation d’énergie. Cependant, la capacité des réseaux va particulièrement être contrainte à courte échéance par la disponibilité du spectre, en raison de la demande conjointe en débit et qualité de service (QoS - Quality of Service) pour un nombre d’utilisateurs toujours plus grand \cite{Cisco_2013}. En outre, il est désormais établi que la question de l’efficacité énergétique va devenir primordiale et qu’il va en résulter un maillage de plus en plus dense de stations de bases (hétérogènes souvent). Par ailleurs, de nouveaux réseaux sans fil vont connaître un développement extraordinaire, tels que ceux qui concerneront l’Internet des Objets (IoT – Internet of Things). On prédit pour l’IoT des milliards d’objets aux capacités radio très différentes, avec des schémas d’accès divers (accès plus ou moins libre, QoS plus ou moins importante) qui vont encore plus densifier l’utilisation du spectre radio fréquence. Ainsi on pressent que doivent émerger de nouvelles solutions d’accès au spectre, plus intelligentes que celles utilisées depuis 100 ans que la radio existe commercialement \cite{Fed_2002}. La radio intelligente (CR – Cognitive Radio) \cite{Palicot_2013}, un nouveau concept apparu à la toute fin du 20ème siècle, étudie et propose des solutions visant à insérer de l’intelligence dans les réseaux et équipements de communications sans fil, afin notamment de répondre à ces défis de rareté spectrale et d’efficacité énergétique qui sont au cœur de cette thèse. En combinant les facultés d’intelligence et de flexibilité, la radio intelligente ouvre la voie à l’auto-adaptation des systèmes de communications sans fil aux changements de leur environnement. Quand il s’agit d’améliorer l’efficacité énergétique et de mieux utiliser le spectre, on parle de radio verte, ou d’éco-radio (Green Radio). 

Les principaux axes d’action de la radio verte peuvent donc être : la gestion du spectre et les réseaux sans fil économes en énergie.

\subsection*{Gestion du spectre}

L’utilisation efficace du spectre radio-fréquence est un problème fondamental des communications sans fil. Les relevés statistiques de l’usage fréquentiel et temporel des fréquences présentés dans \cite{Fed_2002, Palicot_2013} établissent que le spectre n’est tout d’abord pas complètement utilisé à tout instant quand il est assigné, et que cela peut fortement dépendre du lieu. D’après la Federal Communications Commission (FCC), cela concerne entre 15\% et 85\% du spectre alloué \cite{FCC_2003}. Qu’une telle proportion du spectre soit sous-utilisée alors que la demande spectrale est toujours plus forte, appuie les arguments des tenants d’une utilisation plus flexible des ressources spectrales, basée sur la radio intelligente.

Les technologies d’accès dynamique au spectre (DSA - Dynamic Spectrum Access) visent à répondre à cette question \cite{QZhao_2007}. L’accès opportuniste au spectre (OSA – Opportunistic Spectrum Access), en particulier, est un cas de DSA pour lequel un utilisateur secondaire (SU) exécute des techniques de détection de présence d’un signal (sensing) et de prise de décision pour accéder aux ressources spectrales quand elles sont laissées vacantes par les utilisateurs primaires (PUs) \cite{Wassim_2010, Grace_2012}. Les SU sont donc dotés de capacités de radio intelligente. Le DSA permet ainsi à ces SUs de s’adapter au trafic fluctuant des PUs, afin de profiter des opportunités laissées par ceux-ci, et au final de mieux utiliser les ressources spectrales. Autrement dit, les SUs peuvent combler les « trous » laissés par les PUs. Plus les SUs sont capables d’anticiper les opportunités grâce à l’apprentissage, meilleure est l’efficacité spectrale au niveau global, et meilleure est l’efficacité énergétique des SUs qui limitent leurs tentatives de transmissions et leurs risques de collision entre eux. 

En OSA, il est primordial que les SUs n’interfèrent pas avec les PUs qui doivent garder la même QoS que s’il n’y avait pas de SUs (conditions d’acceptation par les possesseurs de bandes licenciées de ce genre d’approche), mais il est aussi important que les SUs aient également une certaine QoS, qui prise au sens large peut consister à ne pas seulement rechercher à utiliser des canaux vacants, mais aussi dans des conditions offrant une certaine qualité. La qualité peut aussi bien concerner le taux d’interférence dans un canal, que la consommation d’énergie que requiert une transmission dans ce canal pour un taux d’erreur donné. Ce sera l’un des objectifs de cette thèse de prendre en considération la qualité dans l’apprentissage de systèmes radio intelligents.



\subsection*{Les réseaux sans fil économes en énergie}

De la croissance permanente de la demande en communications découle une augmentation régulière des émissions de CO2, issue de la consommation électrique des points d’accès radio, des routeurs du réseau, ainsi que des centres de calcul et de stockage. Ils constituent les principaux consommateurs d’énergie de l’industrie des technologies de l’information et des communications (TIC) qui représentait il y a encore peu de temps 2\%, et bientôt 10\%, de la consommation énergétique mondiale \cite{Ajmone_2009}. Dans les réseaux mobiles, les stations de base représentent 60 à 80\% de la consommation totale \cite{Fettweis_2008} et les opérateurs doivent régler une facture annuelle de plus de 10 milliards de dollars en consommation électrique \cite{Son_2011,Peng_2011}. Il y a par conséquent de forts intérêts économiques et écologiques à prendre en considération l’efficacité énergétique dans les réseaux de communication sans fil. Il est important de constater que ces réseaux sont actuellement dimensionnés pour fonctionner en permanence au pire cas correspondant aux pics de trafic auquel ils ne doivent faire face que de temps en temps. Ainsi en raison des fluctuations du trafic dans le temps et du déplacement des utilisateurs, le réseau est par moments surdimensionné par rapport à la demande instantanée, et par conséquent sa consommation ne diminue pas pour autant en proportion du nombre d’utilisateurs connectés. C’est tout l’enjeu du projet TEPN (Towards Energy Proprtionnal Networks) du Labex Cominlabs dans lequel se situe cette étude. En outre, on remarque qu’il serait possible par moment de mettre en veille certaines stations de base, dont le ratio entre la puissance radiative effectivement émise par son antenne et la puissance totale consommée de la station de base (incluant tous les traitements numériques mais aussi l’air conditionné du local) n’atteint péniblement que 3\% \cite{karl_2003}. 
%Des études récentes sur le trafic temporel, comme le montre la Fig. \ref{fig:Traffic_variation1}, indiquent que les stations de base sont largement sous-utilisées. 
Ainsi certaines études ont montré que les stations de base sont très souvent en sous-charge, et que pendant 30\% du temps en semaine (45\% les week-ends), la charge est inférieure à 10\% de la charge maximale que la station de base peut supporter, en termes de trafic et de nombre d’utilisateurs \cite{Oh_2011}. Durant ces périodes de faible charge il a été montré \cite{Son_2011} qu’il est particulièrement économe d’éteindre ou mettre en veille certaines des stations de base et de déporter le faible trafic qu’elles devraient supporter sur leurs voisines, elles-mêmes en sous charge.
%\begin{figure}
%\centering
%\includegraphics[width=0.8\textwidth, height=0.5\textwidth]{./Chapter2_Figures/Normalized_traffic_load.png}
%\caption{Charge de trafic réel normalisée pendant une semaine qui est enregistrée par l'opérateur cellulaire. Les données captent les informations d'appel vocal sur une semaine avec une résolution d'une seconde dans une zone urbaine métropolitaine et sont calculées sur une échelle de 30 minutes \cite{Oh_2013}.}
%\label{fig:Traffic_variation1}
%\end{figure}

C’est le but de la planification dynamique des réseaux, qui vise à contrôler le nombre de stations actives en fonction du trafic. Le projet FP7 EARTH illustre des cas d’utilisation pour le LTE \cite{FP7EARTH}. Un enjeu important dans ce cas, et c’est ce que propose d’étudier les présents travaux, est de définir des stratégies de prise de décision, basées sur de l’apprentissage, pour contrôler le nombre de stations de base qui doivent être laissées en fonctionnement afin de maintenir un service adéquat (avec une certaine QoS).

\subsection*{Plan de la thèse et contributions}

Nous montrons dans ce manuscrit que l’apprentissage et la prise de décision sont matures pour un déploiement réel, en termes de rapidité de convergence, de complexité de mise en œuvre et de performance, à la fois dans le cas d’un terminal seul et dans le cas de nombreux terminaux en réseau. Nous considérons les deux cas de l’efficacité de l’utilisation des ressources spectrales et de l’efficacité énergétique.

Après un chapitre introductif, le \textbf{chapitre 2} du manuscrit effectue un état de l’art des travaux relatifs au sujet de notre étude. Dans un premier temps une analyse de la littérature de la prise de décision pour la radio intelligente dresse la liste des contraintes associées au problème posé, et identifie les solutions potentielles d’apprentissage machine, notamment l’apprentissage par renforcement. L’analyse a permis de déduire que le modèle des bandits manchots (MAB – Multi-Armed Bandit) et les algorithmes qui y répondent sont adaptés à la prise de décision pour la gestion du spectre et l’efficacité énergétique des réseaux sans fil.
Les principales contributions de cette thèse se trouvent dans les chapitres 3 à 6. 

Le \textbf{chapitre 3} introduit un nouveau critère de caractérisation de l’efficacité potentielle de l’apprentissage, le facteur OI (Optimal Identification). Ce facteur va être utilisé ensuite pour évaluer les performances de l’apprentissage de manière plus objective. Il va notamment permettre de pouvoir mieux comparer les performances apportées par l’apprentissage par rapport à un comportement aléatoire, mais aussi de caractériser un scénario en termes d’honnêteté : est-ce un scenario facile ou difficile pour l'apprentissage?

Le \textbf{chapitre 4} introduit un nouvel algorithme d’apprentissage dans le cadre de l’accès opportuniste au spectre (mais qui peut s’appliquer au-delà), qui combine deux critères : la disponibilité des canaux fréquentiels et un critère de qualité (par exemple le rapport signal à bruit sur interférence des canaux). Cet algorithme est basé sur l’extension des précédents travaux ne prenant en compte que la disponibilité des canaux avec des algorithmes de type UCB (Upper Confidence Bound) \cite{Wassim_2010} avec une approche de modélisation du problème MAB par chaines de Markov, dans un cas « restless » : d’où son nom « restless Quality of Service UCB » ou RQoS-UCB. Dans ce modèle, deux récompenses sont prises en compte (disponibilité et qualité) et combinées pour hiérarchiser les solutions entre elles et permettre à un SU d’optimiser la sélection des canaux dans un cas OSA (et tout autre problème répondant au même modèle). La preuve analytique de convergence, c’est-à-dire une borne supérieure de la récompense d’ordre logarithmique dans les cas de chaines de Markov « rested » et « restless », est donnée tout d’abord dans le cas d’un seul joueur, pour valider l’approche, puis dans le cas de multi-joueurs non coordonnés. Il est montré par des simulations que l’algorithme RQoS-UCB proposé permet l’optimisation de l’exploitation d’opportunités de transmission minimisant les collisions entre des SUs alors que ceux-ci ne sont pas coordonnés, c’est-à-dire qu’ils n’échangent pas d’information entre eux (ce qui consommerait de manière pénalisante une partie significative de la bande passante qu’ils essaient justement de trouver de manière opportuniste). \textbf{Les principales contributions du chapitre 4 sont} :
\begin{itemize}
\item La modélisation du scenario OSA sous la forme de problèmes MAB markoviens « rested » et « restless » prenant en compte deux critères (qualité et disponibilité) pour faire l’apprentissage des opportunités de transmission parmi plusieurs canaux, et permettant de sélectionner à chaque instant le meilleur canal.
\item Un nouvel algorithme RQoS-UCB mono-joueur et la démonstration mathématique que la récompense a une borne supérieure d’ordre logarithmique dans le cas markovien « restless ».
\item L’extension de l’utilisation de l’algorithme RQoS-UCB au cas multi-joueurs et sa capacité à permettre aux joueurs d’éviter les collisions de manière non coordonnée.
\item Une validation par des simulations (et dans un prochain chapitre par des démonstrations sur signaux radio réels) de nombreux scenari qui valident l’approche proposée et montrent qu’elle surpasse les solutions de l’état de l’art.
\end{itemize}

Le \textbf{chapitre 5} vise à utiliser l’algorithme RQoS-UCB dans un autre contexte que l’OSA, celui de l’efficacité énergétique des réseaux, avec des stations de base (homogènes ou hétérogènes) pouvant être mises en veille. Ce problème est en phase avec le projet TEPN (Towards Energy Proprtionnal Networks). Le problème de commutation dynamique de l’état (allumé, veille) des stations de base est modélisé par une approche de type MAB markovien. Les performances de l’algorithme RQoS-UCB sont évaluées puis une évolution basée sur le transfert de connaissance (TL - Transfer Learning) est introduite pour le cas des problèmes MAB : TRQoS-UCB. L’analyse de la preuve de convergence de la solution proposée est très similaire à celle de l'algorithme RQoS-UCB et des résultats de simulation mettent en valeur les gains obtenus en termes d’efficacité énergétique. \textbf{Les principales contributions du chapitre 5 sont} :
\begin{itemize}
\item La modélisation du problème de commutation des stations de base en un problème MAB markovien et l’utilisation de l’algorithme RQoS-UCB pour le résoudre.
\item L’utilisation du principe de transfert d’apprentissage pour initialiser les algorithmes de prise de décision du problème MAB.
\item L’évaluation de ces principes par des simulations pour l’efficacité énergétique des réseaux sans fil.
\end{itemize}

Dans le \textbf{chapitre 6} sont présentées les preuves de concept réalisées lors de ces travaux de recherche afin de démontrer la faisabilité et la pertinence des approches proposées. Tout d’abord, l’approche OSA a été appliquée au canal de transmission HF (Hautes Fréquences) qui est un canal de communication transhorizon utilisé par les radio amateurs et les militaires pour communiquer à l’échelle du globe, en profitant de phénomènes de propagation particuliers intervenant dans la gamme des fréquences HF (3 MHZ-30 MHz). Comme ces communications sont naturellement trans-frontalières, de nombreuses collisions peuvent intervenir entre les utilisateurs et il n’existe pas à l’heure actuelle de solution de coordination pour les mitiger. Des algorithmes d’apprentissage UCB ont été évalué sur des bases de données de mesures réelles issues de canaux HF, effectuées par l’Université de Las Palmas de Gran Canaria, et les résultats valident la solution proposée. Ensuite des preuves de concept du cas d’accès opportuniste au spectre ont été menées sur des signaux radio réels émis en laboratoire. Plusieurs algorithmes de bandits (UCB, Thomson Sampling, KL-UCB, QoS-UCB) sont comparés en temps-réel, dans différentes conditions de comportement du réseau primaire (i.i.d., markovien, pour différents schémas d’occupation). Le cas mono-joueur permet de bien comprendre le gain apporté par l’apprentissage, d’évaluer sa pertinence en termes d’efficacité (pourcentage de succès de transmission opportuniste) et de vitesse de convergence et de comparer les différents algorithmes entre eux, en termes de coût de commutation. Ensuite, les démonstrations multi-joueurs permettent là encore de valider la pertinence des propositions faites dans cette thèse, notamment en mesurant le taux de collisions entre utilisateurs secondaires qui sans se coordonner (aucun échange de message) arrivent à se répartir sur les canaux de manière avantageuse pour l’ensemble des utilisateurs.

Enfin le \textbf{chapitre 7} conclut cette thèse et propose des perspectives aux travaux de recherche qui y ont été menés. Ces travaux ont abouti à la production de 3 revues internationales dont deux IEEE Transactions on Cognitive Communications and Networking, un brevet, 6 conférences internationales et deux conférences nationales. 

