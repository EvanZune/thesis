% Chapter 2

\chapter{Real Data Measurements in Cellular Networks: Clustering is Everywhere} % Main chapter title
\minitoc
\label{Chapter2} % For referencing the chapter elsewhere, use \ref{Chapter1}

\lhead{Chapter 2. \emph{Real Data Measurement in Cellular Networks}}
% This is for the header on each page - perhaps a shortened title

%----------------------------------------------------------------------------------------
This Chapter will give a comprehensive review on the real data measurements in cellular networks, including mobile users, traffic demand, and base stations. According to the corresponding characteristics of each measured subject, the introduction is presented on different dimensions, such as temporal dynamics, spatial distributions or content classification.

\section{Introduction}
After the depiction of the network architecture, this section will display a broad picture of the real data measurements in cellular networks firstly, focusing on explaining why this field rises as a popular research topic, how important is the real data for the cellular network research and what will be its benefits for the performance analysis.

\subsection{Cellular Networks Architecture}
Since the invention, cellular networks have been designed as a modular architecture that allows inter-operability among different generations across diverse technologies and distinct requirements. During the past decades, from a functional point of view, the overall structure of cellular networks has been unchanged and physical entities remain grouped into two domains: RAN (Radio Access Network) and CN (Core Network) domains. The RAN domain provides users with radio resources to access the CN domain (uplink or downlink), while the latter is responsible for the management of services, including the establishment, termination and QoS based parameter reconfiguration. Fig. \ref{networks_architecture} outlines these domains across 2G, 3G, and 4G networks according to the corresponding major standards, i.e., the GSM (Global System for Mobile communications), the  UMTS (Universal Mobile Telecommunications System) and the LTE (Long-Term Evolution), respectively.

\subsubsection*{2G GSM Networks}
The RAN domain in 2G networks is named as Base Station Subsystem (BSS). It consists of Base Transceiving Stations (BTS) and Base Station Controllers (BSC). As BTS is responsible for radio transmissions and receptions along with some physical layer processing, a BSC is in charge of a group of BTSs. Moreover, BSC is responsible for the management of radio resources, paging and handover procedures under its coverage area. On the other hand, the CN domain referred to as Network and Switching Subsystem (NSS), only performs circuit-switched function, and it is usually formed by Mobile Switching Centers (MSC) and Gateway Mobile Switching Centers which is responsible for voice call control, user equipment (UE) registrations and mobility management. Besides, several major databases useful for managing customers are also included in the CN domain, i.e., the Home Location Register (HLR) which stores the detailed description of registered subscribers, the Visitor Location Register which helps for the roaming precedure, the Authentication Center dealing with the authentication and encryption procedures, and the Equipment Identity Register storing the unique identification of each subscriber.

\subsubsection*{3G UMTS Networks}
The Universal Terrestrial Radio Access Network (UTRAN) represents the RAN part in UMTS networks and it is composed of NodeBs and Radio Network Controllers (RNC) that match up with BTSs and BSCs in GSM networks. On the other hand, the CN, is divided into two parts: the circuit-switched and packet-switched domains, representing practically a combination of the GSM NSS, and the General Packet Radio Service (GPRS) backbone. We remark that GPRS is a technology between 2G and 3G cellular networks, that provides mobile data services with data rates of a few Kb\/s. The PS domain consists of the Serving GPRS Support Node (SGSN) and the Gateway GPRS support node (GGSN), responsible for handling packet connections of UEs, security functionalities and mobility management functions as well as data routing.

\subsubsection*{4G LTE Networks}
In contrast to UMTS systems, LTE networks are designed to provide only PS services. The RAN, or so-called Enhanced-UTRAN in 4G era, is only formed by interconnected base stations called eNodeBs, without centralized controlling entities, which is opposed to its preceding technologies. Similarly, the eNodeB takes responsibility for radio-related functions and it is directly connected to the core network, which is referred to as Enhanced Packet Core (EPC). The EPC, that is responsible for the overall control of UEs, includes Serving Gateways (SGW), Packet Gateways (PGW) who manages data packets routing and forwarding, as well as network address allocations, and Mobility Management Entities (MME) performing connection management. Additionally, by cooperating with the following other entities: Home Subscriber Server, Enhanced Serving Mobile Location Center, and Gateway Mobile Location Center, the MME completes mobility-related and authentication-related tasks. Finally, the EPC also includes a Policy Control and Charging Rules Function entity orchestrating policy and how control decision makings.

For the billing and inter-operator accounting procedures, a set of logical charging functions are implemented in the network. Specifically, these elements collect network resource usages of each customer and implement following functions: the Charging Trigger Function, which generates charging events based on the observation of network resource usages; the Charging Data Function, which receives charging events from the CTF to construct Call Detail Records (CDR), providing for each user reports concerning his communications; and the Charging Gateway Function, responsible for validating, reformatting and storing CDRs before sending them to the billing domain \cite{rappaport1996wireless}.

After introducing the overall architecture of cellular networks briefly, following subsections will discuss the real data measurement and underlying applications in literature.

\begin{figure}[!htb]
\centering
\includegraphics[trim=0mm 0mm 0mm 0mm,clip,width=0.4\textwidth]{./Chapter2_Figures/networks_architecture.png}
\centering
\caption{The architecture of typical cellular networks with different generations of technologies.}
\label{networks_architecture}
\end{figure}

\subsection{Real Data as a Source}
% discovering internal patterns of human dynamics
% detecting the technical flaws of cellular networks
As a result of the rapid development during the last decade, the cellular networks is becoming more and more complicated, as introduced in the last subsection. Therefore, in order to make an insight view into the internal structure, real data measurement emerges as an accurate and inexhaustible way as database and processing technologies keep upgrading.

Besides, owing to the overwhelming popularity of smart phones around these years, the cellular networks are able to provide an all-weather recording of nearly every person around the world including the locations and communication activities \cite{calabrese2011real} which are usually used for billing purposes only. From this point of view, the real data from global cellular networks may not only provide insights into the system itself, but also make benefits for many other fields, such as human dynamics \cite{song2010limits} \cite{trasarti2015discovering}, city planning \cite{ratti2006mobile}, traffic prediction \cite{calabrese2011real} and so on. Regarding to the related progressive research on this topic, cellular networks actually have been the main contributor of real data worldwide and is promising to reach a general data-fetching platform for inter-field scientific collaboration.

Return to the cellular network analysis itself, actually, the telecommunication operators have been volunteers long time ago to use real data from their own networks to improve the service quality. For example, based on the signal strength across the coverage area and the outage probability across different cells, the operators decide where to deploy the new base stations and which is the most appropriate transmit power. However, these kinds of usage that lacks overall planning and fundamental principle are not good enough to make full utilization of real data from cellular networks. As an example, authors in \cite{rosen2014discovering} made use of worldwide real data consists of fine-grained RRC (Radio Resource Control) dynamics and provided a detailed analysis of the RRC inter-state timeout setting which significantly contributes to the average latency of user data transmission.

Conclusively, real data in cellular networks can not only be a reliable and active source for problem detection in the technical point of view, but also provides a accurate and comprehensive source for pattern discovery in the theoretical point of view.

\subsection{Real Data as a Standard}
% evaluating whether a specific technology works
% reaching accurate parameters for different models
Driven by the ever-growing traffic volume and high QoS requirement in cellular networks, a lot of new architectures and specific technologies have been proposed in recent years claiming that each of these technics provides more or less performance improvement. However, many propositions of algorithms or methods are merely based on theoretical simulation but do not involve any real scenario examination.

Firstly, the performance evaluation solely based on the simulation results is not a wise choice for real network deployment, because it probably underestimates the complexity of actual communication environment. Furthermore, the scale of applicability for a specific newly proposed technique is difficult to predefined, i.e., maybe it's applicable in a single cell but causes unforeseen severe performance on the system level. Therefore, it's vital to enroll the real data identification process for all newly proposed technics to make sure the potential adoption will not cause chaos.

Secondly, usually most of the parameters in a simulation are set according to some traditional statistical assumptions which are summarized from the real measurements long time ago, such as the path loss models in wireless links and the Poisson arrival of user requests. However, these kind of assumption need to be verified or adjusted by new measurements from ongoing networks because of the rapid change of communication environment and human dynamics \cite{steenbruggen2013mobile}.

In this point of view, real data should play a standard role in the cellular networks analysis. Besides, unlike decades ago when large scale of measurements are very difficult to be conducted, nowadays' data measuring and storing technologies are advanced enough to record every detail of the activities within the cellular networks system. Therefore, real data is able to function as a standard to evaluate the various proposed techniques.

\subsection{Real Data as a Solution}
% predicting tendencies, load, functions of human-oriented systems
% reflecting online network status, service, traffic load
In fact, cellular networks are highly related to other human-related systems, like the transport system, the social networks and the residential planning. Commonly, these kind of systems all stem from the daily dynamics of human beings thus showing more or less similarity with each other. In this point of view, real data measurement from cellular networks can be beneficial to solve strategic problems in other fields, such as the transportation traffic prediction or the information gossip spreading.

In detail, the snapshot record of the cellular networks provides a density map of the mobile users within the coverage area, which is able to characterize the spatial distribution of corresponding transportation \cite{caceres2012traffic}. Therefore, combining the continuous snapshots description, the real data can depict the movement of mobile users, so are the related transport traffic which is really helpful for the transportation planning.

Also for the social networks, cellular networks measurement is able to provide a holistic view of how the mobile users communicate with each other, thus tracing the link flows which supports the specific information gossip. Even more, based on real time real data, the monitoring system is able to detect the vital node who plays the key role in the link flows, and necessary steps can be implemented to cut or weaken the information gossip.

Considering the universality of cellular networks and the possibly detailed records, real data measurement is able to provide a effective solution for many other human-related systems. In \cite{shafiq2014understanding}, the author builded a clear correlation between the SINR level in cellular networks and the user engagement rate of video websites, giving specific suggestions for OTT (Over the Top) providers to improve the user experience in the network point of view. Return to the cellular networks themselves, real data is also kind of solution for the more and more important traffic prediction problem.

Actually, traffic prediction is crucial in cellular networks because its accuracy impacts the allocation of scarce communication resource \cite{tan2008empirical}, especially the frequency resource. Practically, historical traffic records in cellular networks can serve as the training set of the predict procedure, along with the increasingly powerful machine learning techniques, higher predicted accuracy can be achieved with larger amount of real data. Therefore, real data is promising to provide effective solutions for problems across from cellular network itself to various related human-oriented systems.

Conclusively, today's cellular networks are already very complicated in its implementation, not to mention the forthcoming 5G networks which is destined to involve multiple access technologies and satisfy multiple classes of service criterions. For simplify the designation of cellular networks and make it more efficient, empirical real data approach might be a promising way. In our claim, real data measurement can be adopted as a source, a standard and a solution for the evolution of cellular networks, which is beyond the subjective choice. After introducing the cellular network architecture and potential applications of real data, we will give a comprehensive review on the real traffic measurement in cellular networks. In this scope, three dimensions of traffic dynamics will be presented, i.e., time, space and content.

\section{Real Traffic Measurement in Cellular Networks}
After explaining the crucial importance of real data measurement in cellular networks, we will introduce the real traffic measurement as a major part of the state of the arts. Cellular networks traffic, along with the development of network infrastructures, is going through a evolutional changing in dimensions of volume, type and demand pattern. As far as in 1997, author of \cite{wirth1997role} indicated the paradigm shift from traditional circuit-switched calls and Erlang-distributed patterns to advanced packet-switched data and various multimedia services and appealed for new teletraffic model and service paradigm. Traffic in cellular networks, including message \cite{zerfos2006study}, voice \cite{willkomm2008primary} and data consumption \cite{shafiq2011characterizing}, has shifted from the voice-domination consuming pattern to the data-domination consuming pattern, resulting from the network development over years. Generally, the traffic as a wide-ranged variable, requires many features or dimensions to characterize, from statistical description to geographical representation. In this section, to be clear and complete, we mainly cover these three most principle features, i.e., the temporal, spatial and content description of traffic in cellular networks. Actually, these three dimensions of characterization provides the most influential impact on the coordination between network function and user demand. As following, we will introduce them one by one in specific details.

\subsection{Temporal Characterization of Cellular Traffic}
Traditionally, the traffic volume is considered to be homogeneous across different time scales for a cell, a BTS or a BSC, in order to simplify the planning procedure of cellular network. However, the absolutely uniform assumption is definitely not true, and thus the desired network capacity is decreased since base stations are usually fuelled with the necessary radio resource to meet the highest traffic demand. On a smaller time scale, it is usually assumed that the call activities and data requests follow a Poisson process temporally, which means that the arrival interval between subsequent calls is exponentially distributed. Actually, this kind of assumption can be true as A. K. Erlang claimed the Poisson distribution of telephone traffic one hundred years ago. However, as the technology evolves from wired traffic to wireless links, and cellular networks embrace multimedia as new approach for human communication, the inherent pattern of traffic demand may also have changed. To identify the validation of traditional Poisson distribution or discover new suitable model, real data is the good and only way to work.

In temporal dimension, there are two different approaches for characterizing the traffic in cellular networks, i.e., the macro view and the micro view. In detail, the macro view means to deal with the summation of traffic volume with a cell, a BS or larger coverage area, and of which the temporal variations on different time scales are analyzed. On the other hand, the micro view includes the traffic records on the levels of users, applications, requests or even packets as data source, and based on which specific performance metrics are investigated on a smaller time scale compared to the macro view. Besides, on the application of different approaches, analytic results from the macro view aim to provide guidance on the overall resource allocation, infrastructure deployment and other macroscopic visions on the cellular network utilization. While, differently, the specific examination of micro view tends to influence the internal parameter tuning or protocol designing for refining the present flaws and improving the microscopic performance. In the following presentation, we will divide the related works into two different categories, according to the macro or micro point of view, and compare them with these correspondingly traditional assumptions.

\subsection*{Macro View}
To analyze the traffic dynamics of cellular networks in a temporally macro view, the aggregation traffic of different cells usually are the crucial variables, and then combined into the total traffic under a coverage region of BSC or MSC. Here, we will introduce the related work according to the research timeline and corresponding logics.

Date back to 1999, Almeida et al. investigated the temporal variation of voice traffic from a GSM network in Lisbon \cite{almeida1999spatial}, where they found a similar temporal pattern (highly related to the daily schedule of local residence) for different cells. Furthermore, they proposed to use the common double-gaussian and trapezoidal distribution to model the temporal variation of voice traffic, and showed that different models apply to different areas. To the best of our knowledge, this work is the first time to characterize the voice traffic of cellular networks in a temporal way, though the data or message traffic are not included.

Regarding the message traffic, in \cite{zerfos2006study}, the authors performed a comprehensive measurement of the SMS (Short Messages Service) in a national cellular networks in India, when the short messages were still popular in 2006. According to their results, 7.2\% of the total messages sent by mobile users are requests to SMS services, and at least 10.1\% of total received messages are sent by content providers instead of mobile users, which is contradictory to our conventional understanding. Although the message traffic is considered in this study, the related temporal variation is not well presented. Accordingly, based on the real measurement from telecommunication operator in China, authors in \cite{zhou2012predictability} showed that the message traffic adopt a periodic pattern on the scale of BSC.

Different with the results in \cite{almeida1999spatial}, authors in \cite{willkomm2008primary} found that the voice load of individual sectors varies significantly even within a few seconds in the worst case and also there exists high variability of traffic volume even across sectors of the same cell. Therefore, the real measurement results tend to diverge in different point of view, on different coverage region. Generally, the larger is the investigated area, the aggregated traffic dynamic tends to be more regular, due to the average effect. For example, \cite{zhou2012predictability} also depicted that the voice traffic within a BSC tends to be significantly periodic, therefore inducing a considerable predictability.

After 2007, as the worldwide spread of smart phones, more and more data traffic are generated, following the decrease of traditional voice and message traffic. Therefore, the temporal analysis of traffic dynamics in cellular networks turns from voice duration to data consumption, which tends to exhibit a similar but more variable pattern.

Based on data set spans one week in 2007 from a nation-wide network with thousands of base stations, authors in \cite{paul2011understanding} showed that the aggregate network load exhibits a nice periodic behavior with relatively high loads during the day and the lowest load during midnight. On the contrary, individual base station loads do not show that much periodicity. Also, the load curve varies significantly among individual base stations with their peaks occurring at different times of the day. Similarly, this study on data traffic shows a regular pattern like the traditional voice traffic, along with the claim that the aggregation traffic in larger area tends to appear more periodicity than individual smaller coverage area.

Utilizing a one-week-long data records in 2010 from a specific state in USA, in \cite{shafiq2011characterizing} authors found a diurnal characteristics of traffic volume over the duration of complete week while weekdays tend to attract more user activity than weekend. Furthermore, the time-series of aggregate Internet traffic volume can be modeled using a multi-order discrete time Markov chain which would contribute to a good analytical property for performance evaluation.

Concluded from these above related studies, the aggregation traffic of large coverage area in cellular networks tend to be significantly periodic which leads to high predictability, while the total traffic in smaller coverage areas (i.e. with a BS) shows different levels of variation which could be attributed to human mobility and geographical diversity. To reach a thorough understanding into the traffic dynamics, more in-depth measurement should be conducted.

Such as in \cite{wang2015understanding}, Wang et al. presented a comprehensive analysis of the data traffic temporal dynamics across thousands of base stations from Shanghai, China. Using machine learning techniques, base stations can be clustered into five different categories based on their traffic dynamics on different time scales, where these categories are highly related to the geographical locations of cellular towers. Besides, they conducted a spectrum analysis on the frequency domain and declared that every traffic from the investigated BSs can be constructed using just four principal components corresponding to human activities.

Also in \cite{wang2015characterizing}, the authors quantitatively characterized the spatio-temporal distribution of mobile traffic based on large-scale data set obtained from 380,000 base stations in Shanghai spanning over one month. They found that the mobile traffic loads uniformly follow a trimodal distribution, which is the combination of compound-exponential, power-law and exponential distributions, in terms of both spatial and temporal dimension with accuracy over 99\%.

Furthermore, authors in \cite{xu2016big} implemented a time series approach to analyze a large amount of traffic data from thousands of cellular tower, and they revealed that the mobile traffic temporal pattern can be divided into two components, regular and random parts. Based on the real data analysis, they discovered a high predictability of the regularity component of the traffic, and demonstrate that the prediction of randomness component of mobile traffic data is impossible.

Conclusively, the traffic dynamics of cellular networks on the macro view exhibit a diurnal pattern which can be adopted as reference for the resource allocation on a large scale. Meanwhile, there still exists evident variation for respective traffic across different cells, suggests that operators should conduct different resource allocation policies for different BSs, not like the one-for-all strategy. Next subsection, we will introduce the traffic dynamics in a micro view.

\subsection*{Micro View}
Different with the macro view for traffic dynamics analysis, the micro view provides a more delicate perspective to characterize the temporal properties, usually making use of user-level, application-level or even packet-level measurement records. Meanwhile, the corresponding traditional assumption on this subject is that the arrival pattern of traffic follows Poisson process in the view of a BS or a specific content. Following, we will introduce the realistic phenomenon discovered by many researchers in this field based on the measurement across the world.

Firstly, for the voice traffic, authors in \cite{guo2007estimate} examined the call duration distribution in GSM cellular networks. The result shows that log-normal distribution of call duration is more precise than the exponential or Erlang distribution in the performance prediction of GSM system. As we know, the log-normal distribution preserves heavy-tailed property at some extent, which reflects the inhomogeneous nature of call durations in GSM cellular networks.

Furthermore, authors in \cite{willkomm2008primary} certified this conclusion through another set of real data. In that study, they presented a large scale characterization of primary users in the cellular spectrum and found that the duration of calls are not exponential in nature and possess significant deviations that make them difficult to model. These two similar results based on different data set reveal the realistic phenomenon which is clearly contrary to the traditional homogeneous assumption.

Secondly, we try to examine the similar property for data traffic. In \cite{williamson2005characterization}, Williamson et. al used fine-grained measurements from an operational CDMA2000 (Code Division Multiple Access) cellular network to characterize the wireless Internet data traffic where the authors revealed different patterns compared to traditional Internet traffic assumption. In detail, the results show that the packet arrival process is non-stationary, and exhibits bursty nature rather than a homogeneous Poisson process. Related with this study, the authors in \cite{zhang2012understanding} compared the wireless traffic with traditional wireline traffic, and found that the data sessions in wireless traffic contain less data in more but shorter flows, which typically consists of smaller packets with burstier arrival patterns. Similarly, these two studies revealed the same inhomogeneous property for data traffic as the voice traffic, both of which are high coupled with the human dynamics in daily life.

Go deep into the relationship between cellular network traffic and human dynamics on temporal view, authors in \cite{xavier2012analyzing} investigated the pattern of call activity during crowded soccer events in Brazil based on the real measurement from the operator, and indicated that the dynamic transition of call activities between different cells can shed light on the inherent pattern of human mobility.

Also in \cite{shafiq2013crowd}, the author presented a first performance characterization of an operational cellular network during crowded events, where the temporal deviation of voice call and data traffic is depicted compared to routine days. Besides the performance characterization during crowded events, the authors also proposed two effective methods to mitigate the severe performance degradation which are verified by simulation of real traces.

On the other hand, to link the personal traffic usage to mobile profiles, authors in \cite{oliveira2015measurement} proposed a framework that automatically categorize mobile users into four different profiles according to their daily traffic usage, and based on which the authors calculated the traffic distribution of different kinds of users and create a traffic generator that captures the realistic usage pattern.

Conclusively, investigating the cellular traffic in the micro view provides insights about how the fine-grained traffic is requested and delivered in the networks. Contrary to the traditional exponential distribution for call durations and poisson process for data traffic arrival, many recent investigations show inhomogeneous nature for cellular traffic on different dimensions.

Overall, the temporal analysis of cellular traffic based on real measurement in both macro and micro views challenges our traditional assumptions on many important patterns, such as the equivalent assumption for all BSs and the homogeneous assumptions for call duration and data arrival. According to these related works, the traffic consumption in cellular networks exhibits a periodic pattern for large coverage area which indicates significant predictability, while the traffic temporal pattern for single cells tends to be various across the cellular networks which urges different policy for distinctive BS. Meanwhile, the voice and data traffic in cellular networks exhibit inhomogeneous nature on different metrics, such as the heavy-tailed property for the call duration and bursty inherence for data request, which would display significant importance in the network performance evaluation.

\subsection{Spatial Characterization of Cellular Traffic}
Besides the temporal variation, the spatial distribution of traffic consumption in cellular networks is far away from the uniform assumption, which is usually adopted in the network simulation of academic research. Actually, due to the mobility feature and aggregated residence of human beings, the mobile users are clusteringly distributed within the whole cellular networks. Thus, the traffic demand of mobile users inevitably varies across the overall coverage area. Specifically, the spatial distribution of traffic density is highly co-located with the human residential hot spots.

In practice, in order to manipulate the limited wireless resource efficiently, the operators need to pre-allocate the frequency bands and corresponding transmit powers to different BSs or cells. Obviously, the resource allocation strategy is necessary to be coincident with the real traffic demand, temporally and spatially. For example, the heavily loaded BSs desire more spectrum and transmit power, while the low-load BSs don't need so much, and the specific quantitative allocation is related to the predicted traffic demand based on historical records. Furthermore, the inter-cell interference in cellular networks is also a challengeable issue, which is highly related to the frequency reuse paradigm and transmit power allocation. Ideally, the frequency reuse factor should be various according to the realistic traffic load, because the interference level is dynamically changed. Therefore, the frequency reuse factor should be smaller in hotspot areas than that in low-load areas.

From these points of view, it's essential to distinguish the realistic spatial distribution of traffic demand from traditional assumptions based on real data measurement. Furthermore, using those empirical results of spatial distribution of traffic demand, it's possible to provide more efficient resource allocation strategies and interference mitigation methods to improve the overall capacity of cellular networks.

Actually, from long time ago, the researchers had found that the spatial distribution of traffic consumption in cellular networks are not uniform. For example, authors in \cite{gotzner1998spatial} investigated the inhomogeneous property of voice traffic in GSM cellular networks, and proposed to use log-normal distribution to model the PDF (Probability Density Function) of traffic volume which can not be rejected in different levels of granularity. Besides, they also demonstrated that there is a distinctive capacity gap between homogeneous and inhomogeneous case, which highly motivated the accurate characterization of traffic distribution in order to provide reasonable guidance for cellular networks planning. This study investigated the numerical distribution of traffic volume density regardless of corresponding correlation between human activities and traffic assumption.

In \cite{almeida1999spatial}, besides the temporal dynamics, the authors discussed the spatial distribution of voice traffic in GSM networks, where they discovered the explicit inhomogeneous nature of traffic distribution and adopted several common functions to model this kind of nonuniform. From their results, farther is the cell away from the city center, smaller is the traffic density for that area, which depicts a clear decaying phenomenon. Specifically, considering the relationship between the distance from the city center and the corresponding traffic density, the exponential model presents the worst performance while pairwise linear model outperforms the other candidates.Furthermore, authors in \cite{tutschku1998spatial} proposed a demand-node generating model for infrastructure deployment in cellular networks, which considers the spatial aggregation effect of voice traffic across the coverage area which is highly correlated with human daily activities.

As the cellular networks upgraded and the traffic usage of mobile users transit from the traditional voice calls to the popular data service, which stimulates the academic group to focus on the data traffic consumption across the whole network. For example, studies in \cite{paul2011understanding} showed that 10\% of the base stations experience roughly about 50-60\% of the aggregate traffic load, which indicates the explicit imbalance of data traffic distribution across cellular networks. Besides, it also showed that less than 10\% of subscribers generate 90\% of the load which reflects the clustering effect of traffic distribution on another dimension. This phenomenon from real measurement clearly demonstrated the heavy-tailed characteristics of data traffic in cellular networks on the BS-level and user-level, which is inherently rooted in human dynamics (similar with Matthew effect in economics and sociology).

More specifically, Laner et. al in \cite{laner2012users} investigated the real data from a HSPA (High-Speed Packet Access) networks in Vienna, and found that the mean throughput of the cells within the peak hour varies over roughly three orders of magnitude. Whereas the 10\% of cells with lowest load have a mean throughput of below 1 kbit/s, the 10\% most loaded cells have a mean throughput above 500 kbit/s. Consequently, the often made assumption of a constant traffic density over a large number of cells is inadequate.

Besides the aggregated traffic volume, Shafiq et. al in \cite{shafiq2012characterizing} analyzed the geospatial dynamics of application usages in a large 3G cellular networks, based on traces from both radio access networks and core networks indicating location information and data delivery details. Based on the application usage calculation on different levels, the authors want to classify a number of BSs into different categories. Counterintuitively, they found that cell clustering results are significantly different for traffic volume in terms of byte, packet, flow count, and unique user count distributions across different geographical regions, and the popularity of different applications significantly varies even within a given neighborhood. In addition to the imbalance effect of the aggregated traffic in terms of byte, these results show that the nonuniform of other statistics also exists and reveals distinct degrees of inhomogeneity.

From the intangible heavy-tailed description to detailed statistical modeling, Lee et. al in \cite{lee2014spatial} found that the cell traffic can be approximated by the Weibull or Gamma distribution; the traffic density can be approximated by the log-normal and Weibull distribution which are all contradicted with traditional uniform assumption. Besides, they also found that there is positive correlation between the traffic volume of different cells which are within some specific distance. More importantly, this study provides the possibility of generating the realistic traffic demand across the whole plane of cellular networks, embedded with the traffic correlation between BSs and without losing the inhomogeneity characteristics.

To consider the mobile user distribution and spatial traffic demand jointly, many researchers started to use fine-grained user distribution information to shed light on the inherent relationship between user and traffic. Ding et. al in \cite{ding2016measurement} discovered that the spatial distribution of subscriber density and average traffic demand can be accurately described by log-normal mixture models, and their product, i.e., traffic density also follows a log-normal mixture distribution spatially. Besides, their extensive analysis gave a precise characterization of base station capacities and clustered all base stations into 6 categories based on subscriber density and average traffic demand. These kind of results may help the operators to allocate their limited resources more efficiently based on the traffic demand categories of different BSs.

On the other hand, to combine the spatial traffic consumption with the mobile user spatial distribution and the infrastructure deployment, authors in \cite{mirahsan2015hethetnets} proposed a tunable statistical model capturing the interconnection of BS spatial deployment and spatial traffic distribution, with only two meaningful parameters. This work provides a convenient way to stimulate a heterogeneous infrastructure deployment and the corresponding spatially heterogeneous distributed traffic demand.

Conclusively, similar with the inhomogeneous distribution of traffic demand on temporal dimension, the aggregated traffic consumption presents heterogeneity across the spatial plane within the cellular networks. Specifically, the time-summation traffic volumes of different BSs exhibit heavy-tailed characteristics according to numerous real measurement verification. This phenomenon indicates the severe imbalance of traffic demand across the whole networks, thus urges the load balance technics or differential resource allocation strategies to improve the overall capacity performance.

Furthermore, compared with the aggregated traffic volume on BS level, the traffic density description which diminishes the area effect provides a more intuitive view into the heterogeneous nature of traffic consumption. Based on different data set from different countries, the state-of-the-art statistical model for the data traffic density in cellular networks is log-normal or Weibull distribution, both of which shows some extent of heavy-tailed property and thus verifies the spatial heterogeneity nature of traffic demand more definitely.

Actually, analyzing the spatial distribution of the traffic volume on the one-dimension numerical statistic is not enough for characterization since it loses explicit location information where the traffic demand actually happens. For example, despite the log-normal distribution may be accurate for traffic density values, but it's not adequate for locating the data requests or the aggregated traffic on user-level or BS-level. One possible way for solving this problem is to spatially modeling the user distribution or BS deployment on two-dimension view, and then feed those information into the traffic density description. In this point of view, the spatial heterogeneities of mobile users, base stations and traffic demand are coupled with each other and should be investigated all together which leads to our works on spatial modeling of base stations in Chapter 3.

\subsection{Content Preference of Cellular Traffic}
Traditionally, before the 3G cellular networks, the available bandwidth is not sufficient for mobile users to fetch contents directly from the Internet. At that time, the main goal of wireless communications were mainly the voice call or short message services between subscribed users. Meanwhile, the wired network was speeding up to provide various content options through the Internet, such as the news, pictures and videos. Looking backward, during the past decades, the provided contents and the connected Internet help each other to spread and enrich, and their combination makes our real life and virtual activities seamlessly merged together. On another point of view, the social pattern of us, human beings in Internet era, reflet itself from the touchable real life onto the virtual binary world. Therefore, the wired Internet or mobile wireless network can be utilized as mirrors to reflect the inherent human dynamics, thanks to their well organized recording and almost ubiquitous coverage.

Firstly, for the web caching requests, Lee et. al in \cite{breslau1999web} first showed that the web requests follow a Zipf-like distribution \cite{newman2005power}. Based on various traces collected independently, the authors introduced a simple model for web requests, where requests are independent and distributed according to a Zipf-like distribution. The results showed that this simple model can explain the asymptotic behavior for three properties that are observed in real web cache traces.

Apart from the traditional web content, the authors in \cite{gill2007youtube} conducted an extensive analysis of the YouTube workload, and found that there are (not surprisingly) many similarities to traditional web and media streaming workloads. For example, since access patterns are strongly correlated with human behaviors, as traffic volumes vary significantly by time-of-day, day-of-week, as well as longer term activities (e.g., academic calendars). Similarly, video files are much larger than files of other types, and some videos are more popular than others.

Similarly, Cha et. al in \cite{cha2007tube}  presented an extensive data-driven analysis on the popularity distribution, popularity evolution, and content duplication of user-generated video contents on the Internet. They studied the nature of the user behavior and identified the key elements that shape the popularity distribution, and it was found that the Pareto phenomenon of content requesting is likely caused by both human similarity and the information filtering technics. Summing up the related works on the web content across the last decade, no matter what kind of them, the contents popularity distribution exhibit similar aggregated feature (heavy-tail distribution), although there are some difference between the requesting pattern.

More comprehensively, \cite{shafiq2016characterizing} present a measurement study of a large commercial CDN (Content Delivery Networks) serving thousands of content providers, and they found that top 1\% content publishers account for up to 60\% of the total request count for both small and large object platforms. Moreover, top 1\% content publishers account for more than 90\% of the total request size for the large object platform, while more than 60\% and 50\% objects in the large and small object platforms, respectively, are requested only once.

Conclusively, the content requests on the Internet have a significant tendency to be aggregated on different scales, such as the content popularity, content provider and content duplication. For the mathematical characterization of the popularity distribution, the Zipf-like models present the best performance.

Although the access technology is far different between the broadband wired Internet and mobile wireless networks, the content requests exhibit similar aggregated properties, specifically after the speed-up of cellular networks. On one hand, for the categories of requested contents, the mobile users are able to access to most of the contents on the Internet, including web pages, audio, video and so on. On the other hand, for each category or in the global view, the popularity of different contents also presents an unbalanced phenomenon.

Such as in \cite{shafiq2011characterizing}, based on large amount of real data from cellular networks, the authors discovered that the distribution of network traffic with respect to both individual devices and constituent applications is highly skewed. Only 5\% of the devices are responsible for 90\% of the total network traffic. Moreover, the top 10\% applications account for more than 99\% of the flows. More specifically, these distributions for the popularity of different ranked contents can be modeled using Zipf-like models, which is the same in wired Internet.

Similarly in \cite{jin2012characterizing}, Lin et. al utilized one month's trace data from a US operator, and characterized the data usage pattern in a large UMTS cellular networks. In accordance with expectation, they found that a few users (top 3\%) consume nearly half of the total data traffic, and exhibit distinctive usage pattern compared to normal users. Furthermore, among these heavy users' usages, a small number of dominant applications make up the majority of data traffic, including mobile video/audio sites, social networks and popular mobile applications.

Authors in \cite{erman2011over} examined the video traffic generated by 3 million users in 2011 across one of the largest 3G cellular networks in US, and they found that video traffic accounts for 30\% of the downstream cellular traffic during the busy hour. Besides, the results also showed that 77\% of the traffic is concentrated in just the top 10 content providers and 24\% of the bytes for progressive downloads requests can be served from cache. Conclusively, the data traffic in cellular networks exhibits significant clustering feature not only on the content categories but also on the user usage level, and the popularity of different contents can be modeled as Zipf's law.

Besides the normal time, the clustering nature of content requests expresses more aggressive during the busy hour. For example, in \cite{erman2013understanding}, the authors investigated the traffic dynamics during the 2013 Super Bowl event in New Orleans based on detailed records from the overlapped LTE networks. From their study, the most popular applications during this tremendous event are web browsing and video streaming, and the most accessed content provider are cloud providers which account for over 22.1\% of the overall traffic. Even more dominant than that in broadband networks, the HTTP (HyperText Transfer Protocol) traffic constitutes more than 90\% of the total multimedia traffic in cellular networks, as depicted in \cite{erman2011over} and \cite{maier2009dominant}. From these observations, we can see that the contents preference phenomenon is tied up with the request procedure, especially on the base of large number of mobile users, no matter in busy hour or normal usage.

Furthermore, after revealing the content preference of mobile users in cellular networks, it's also important to investigate the predictability of different kinds of contents and their performance impact on network utilization. Authors in \cite{zhou2012predictability} collected large amount of real data from a GSM/UMTS hybrid cellular networks, and investigated the predictability of three kinds of traffic, i.e., message, voice and data. According to their results, voice traffic has the highest predictability among these three, while nearby BSs and historical records can improve the traffic predictability significantly. Besides, in \cite{zhang2012understanding}, the authors examined a wide range of services in cellular network traffic, and find that different applications impose different demands on network resources on packet level, flow level and session level.

Conclusively, the user requests in cellular networks tend to exhibit content preferences, showing explicit heavy-tailed distribution on content popularity. Together with the spatial and temporal dimensions, the dimension of content forms the complete clustering nature of traffic distribution in cellular networks. After introducing the separate characterization of each dimension, we will investigate the traffic dynamics on any two combination of these three dimensions, and shed light on the relationship between the clusterings on the different dimensions and their potential application on the service capacity optimization in cellular networks.

\section{Clustering Nature of Cellular Networks}
After introducing the real measurement of traffic demand in cellular networks in three different dimensions separately, in this section we try to extend the analysis to the combination of different dimensions. Other than theoretical characterization, description of the combination of clustering nature on different dimensions can lead to potential technic solutions. For instance, combining the spatial and temporal characterization, it's possible to introduce the sleeping mode into the BSs, where nearby BSs need to provide additional coverage to ensure the connection of the users under the switch-off BSs. By making use of the combined clustering property, potential benefit can be obtained by mutual compensation while performing discriminate treatment on different dimensions.
\subsection{Spatial-Temporal Characterization}
In cellular networks, clustering properties are not only exhibited in the spatial distribution, but also on temporal dimension, as we illustrated in previous section. Whether there is a dependence between these two dimensions is crucial for combining them together for more efficient service policies. Hereafter, we try to uncover the possible correlation between spatial distribution and temporal distribution of traffic demand in cellular networks.

Firstly, we need to figure out what is spatial-temporal clustering. In part of it, spatial clustering means that the traffic volume varies in different locations, and small portion of BSs occupy most of the overall traffic. For the second half, temporal clustering means that the traffic demand varies during different time, implying peak and valley, even though exhibiting some extent of predictability \cite{zhou2012predictability}. Combining them together, the spatial dimension of traffic demand may exhibit distinctive degrees of clustering effect at different time period, such as office time and dinner time. On the other hand, the temporal dimension of traffic demand may exhibit distinctive degrees of clustering effect at different locations, such as residence and offices. Therefore, it can be declared that spatial dimension is coupled with temporal dimension showing a twisted clustering phenomenon.

Secondly, how to deal with the correlation between spatial and temporal clustering? Generally, it's a mathematical problem dealing with two random variables. Therefore, we need to separately define the degree of spatial clustering and temporal clustering mathematically, then we can obtain two series of random variables. Finally, we can conduct the formal coefficient calculation for these two series, and obtain a number indicating the degree of correlation of them. Conclusively, the correlation analyzing question reduces to a definition problem, which is necessary to characterize the clustering degree of different dimensions.

Finally, how to make use of this correlated phenomenon? As we showed by empirical results, there are clustering features in spatial and temporal dimensions, and there is correlation between these two dimensions. However, how can we make use of them is still present to solve. Intuitively, in cellular networks, we can switch off some BSs in low traffic region when the spatial clustering effect is quite obvious, while ensuring the coverage function through nearby BSs. However, during this operation, it's necessary that the overall traffic volume is low in order to make sure that nearby BSs are capable of delivering additional capacity. That's to say, the coordination between spatial and temporal is essential for BS-switch operation in cellular networks.

Conclusively, spatial clustering of traffic demand is coupled with that of temporal dimension, and this feature are promising for more energy efficient operation in cellular networks. In following chapters, we will deal with the concrete description of these features and trying to make fully use of them.
\subsection{Spatial and Content Combination}
Besides space and time records, content is another essential dimension for traffic demand, and it's getting more and more important regarding the increasing research of content centric networking. Once upon a time, it's not welcomed that users tend to request for the same content, because it increases the average delivery time according to queuing theory, especially for client-server paradigm. After that, due to development of distributed sharing system and CDN technology, it becomes a good sign for service management. That's to say, the more concentrated is the user content requests, the more efficient can the delivery process be. However, when it comes to the combination of content preference and spatial clustering, what is the effect on service capacity?

Firstly, what is the meaning of combining spatial and content dimension of traffic demand? Like the case for spatial and temporal clustering, we need to figure out what is the meaning of clustering on both dimensions separately. As we addressed, spatial clustering means the imbalance on the traffic distribution at different locations, and content clustering means that most of the user requests are directed to small amount of contents. After that, is there any correlation between spatial clustering and content preference? Or to say, for different locations, is the corresponding popular content also differs from each other? This claim need to be verified by real measurement.

Secondly, how to deal with the combination of spatial and content dimension? If the content preference varies across the cellular networks, then how to characterize it mathematically? As we know, the content preference are characterized by the Zipf's law separately, and the degree of clustering is then indicated by the exponent of the popularity distribution $\gamma$. Here, after introducing the spatial disparity, we then obtain a series of $\gamma$. From this series, we can get a parameter representing the variation of content preference across the whole networks.

Finally, how to make benefits through compensation between content and spatial clustering? For example, the caching technology are popular research topics in cellular networks in recent years, especially for the RAN caching. In RAN caching, BSs are assumed to cache limited amount of content in their local storage, and the contents can be directly delivered to mobile users if requested. In this scenario, spatial clustering of BS deployment may be beneficial since the BSs nearby can perform as a cluster in order to enlarge the caching storage, thus storing more popular contents.
\subsection{Temporal and Content Combination}
Like the combination of content preference and spatial clustering, here temporal distribution is replacing spatial distribution. As described in previous subsection, the clustering preference of traffic demand varies across the cellular networks, and it also changes along the time dimension. Therefore, we need to consider it together in order to track the variation of content preference in a time series way.

Firstly, consider the clustering property of content preference and temporal distribution separately. As described previously, the degree of content preference in cellular networks are well characterized by the exponent of Zipf's law (i.e., $\gamma$). On the other hand, there is plenty of traditional ways to analyze a time series which is consisted by the Zipf's exponents here. However, there is some information lost here, since the exponent is just dealing with the popularity proportion, thus can not tell the order of exact popular contents. In order to fix that, we need to add some other assumptions. Therefore, in this thesis we assume that the content library and the absolute order of contents are all fixed, and the popularity distribution is our main concern.

Secondly, how to combine the analysis of temporal clustering and content preference together? Not only content popularity are various along the time dimension, but also the traffic summation of content requests are dynamic. Therefore, consider these two effects together, the traffic demand for each content would be tremendously various on the time scale. Similar with the combination of spatial and content clustering, we try to use the joint probability to characterize the merge of temporal clustering and content preference.

Finally, how to make use of the combination of temporal clustering and content preference? As we know, the static content preference properties are beneficial for the traffic management in distributed systems, where caching technology can play a crucial role. After combining on time dimension, if the traffic demand of a specific content achieves a predefined threshold, we can adopt broadcasting to perform `one transmit all receive" paradigm. Due to the openness nature of wireless communications, the broadcasting technic are beneficial for improving the service capacity.

\section{Conclusion}
After introducing the related real measurement works in different literature, we can conclude that the clustering nature is widespread in cellular networks, spanning from user traffic to infrastructure deployment. In this chapter, we carefully examine the statistical features of traffic demand, from temporal characterization to spatial distribution, and then the content preference description. For different dimensions, we presents plenty of related works which makes use of real data from all over the world, thus demonstrating a comprehensive picture on the realistic scenarios in cellular networks. Accordingly, we divide the conclusion for the measurement results into three parts, namely temporal, spatial and content preference.

Firstly, the temporal analysis of cellular traffic based on real measurement in both macro and micro views challenges our traditional assumptions on many important patterns, such as the equivalent assumption for all BSs and the homogeneous assumptions for call duration and data arrival. Specifically, the traffic consumption in cellular networks exhibits a periodic pattern for large coverage area which indicates significant predictability, while the traffic temporal pattern for single cells tends to be various across the cellular networks which urges different policy for distinctive BS. Meanwhile, the voice and data traffic in cellular networks exhibit inhomogeneous nature on different metrics, such as the heavy-tailed property for the call duration and bursty inherence for data request, which would display significant importance in the network performance evaluation.

Secondly, similar with the inhomogeneous distribution of traffic demand on temporal dimension, the aggregated traffic consumption presents heterogeneity across the spatial plane within the cellular networks. Specifically, the time-summation traffic volumes of different BSs exhibit heavy-tailed characteristics according to numerous real measurement verification. This phenomenon indicates the severe imbalance of traffic demand across the whole networks, thus urges the load balance technics or differential resource allocation strategies to improve the overall capacity performance.

Thirdly, the user requests in cellular networks tend to exhibit content preferences, showing explicit heavy-tailed distribution on content popularity. Together with the spatial and temporal dimensions, the dimension of content forms the complete clustering nature of traffic distribution in cellular networks. After introducing the separate characterization of each dimension, we will investigate the traffic dynamics on any two combination of these three dimensions, and shed light on the relationship between the clusterings on the different dimensions and their potential application on the service capacity optimization in cellular networks.

Finally, by combining these different dimensions of clustering properties together, it's possible to make better use of the limited resources in cellular networks. In following chapters, after examining the clustering nature more specifically and describe them in a mathematical way, we try to introduce promising technics to improve the efficiency of the whole system.
