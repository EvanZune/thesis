% Chapter 1

\chapter{Introduction} % Main chapter title

\minitoc

\label{Chapter1} % For referencing the chapter elsewhere, use \ref{Chapter1}

\lhead{Chapter 1. \emph{Introduction}} % This is for the header on each page - perhaps a shortened title

%----------------------------------------------------------------------------------------


%----------------------------------------------------------------------------------------
\section{Background}

With the world-spread popularity of smart devices, including smart phones, tablets and numerous virtual reality devices, the mobile user traffic demand is increasing exponentially, unlike the traditional voice service generations ago. According to the latest Cisco VNI (Visual Networking Index) white paper, there will be another 8-fold increase of mobile traffic (i.e., from 3.7 to 30.6 exabytes per month from 2015 to 2020) \cite{vni2016}. Driven by these tremendous data consumption and higher QoS (quality of service) requirement, the research groups, including academics and industrials, devote themselves to upgrade the cellular networks by looking into the user demand dynamics and proposing brand new technologies.

On one hand, with the amazing development of the big data technology these years, operators are coming to realize the increasing importance of the recorded data from their own networks. Thus, more and more real data are put into investigation by the operators themselves or contributed to related researchers aiming to reveal the inherit pattern of mobile user activities and mobile traffic distribution. On the other hand, along with the 3GPP (3rd Generation Partnership Project) releases, many researchers are working on new architecture or new algorithms to increase the network capacity and improving the QoE (Quality of Experience) of mobile subscribers. However, the effectiveness or efficiency of these new kinds of technologies in cellular networks need to be verified based on large scale of real data measurement.

To make full use of these methodologies of two different directions, the best way is to combine these two method together smoothly. That's to say, based on the real data examination part, large-scale identification can give rise to accurate and reasonable characterizing models which can be adopted in the technology verification part. Actually, the investigated topics of cellular network real data include traffic, BSs (Base Stations) and mobile users, along with their spatial and temporal distributions. Therefore, these realistic mathematical model can be adopted as preliminary assumptions for academical and industrial research, such as the recently popular stochastic geometry domain which is based on the homogeneous PPP (Poisson Point Process) distribution of base stations or mobile users.

Furthermore, besides the verification benefits of realistic network models, what makes the real data so important is that the changes of users' patterns or the so-called evolution of cellular networks can only be reflected by the network measurements. For example, the exponential distribution is usually adopted as the arrival rate pattern of user traffic in 3GPP's protocol design. However, due to the explosively growing user number and traffic amount these years, the burst nature of network phenomenon is becoming more and more universal which makes the exponential assumption not reliable any more.

\section{Research Topics}
In this thesis, firstly in Chapter \ref{Chapter2}, we present a comprehensive review on the statistical characteristics of traffic demand in cellular networks on different dimensions (space, time and content). In detail,  we review the state-of-art temporal analysis of cellular traffic consumption, on both macro view (aggregated traffic on cells or upper level) and micro view (individual traffic of user or application level). According to the related works, the traffic demand in cellular networks exhibits universal clustering nature on different time scale and on different aggregation level. In parallel, we also introduce the spatial examination of traffic demand, and it also expresses significant level of aggregation effect. For example, the urban and rural areas in large scale and the cell-center and cell-edge part within a BS coverage area. At last, besides the temporal and spatial dimension, we also find the heavy-tailed phenomenon of traffic demand on content dimension. That's to say, mobile users tend to request the same content and the popularity distribution of users is distinctly unbalanced. Furthermore, it's important to know that the statistical features of cellular traffic are basically caused by the usage pattern of mobile users, on all three different dimensions. Therefore, we can conclude that the mobile users in cellular networks are also clusteringly distributed, their temporal usages of traffic are also aggregated and their content preferences are more or less similarly concentrated.

After that, based on large-scale real data from on-operating cellular networks, focused on spatial domain, we investigate the spatial distribution of base stations. In detail, the locations of base stations are examined firstly to check the traditional assumptions, such as the hexagonal placement and homogeneous Poisson point processes. According to the fitting results, the traditional widely adopted spatial models are inappropriate for the real deployment. Therefore, we conduct a large-scale comprehensive identification of base station spatial distribution in cellular networks, considering various of popular two-dimensional point processes including repulsive and attractive ones, and try to figure out the most accurate spatial model for both macro and micro BSs. Unfortunately, based on the large amount of real data, all of these well understood point processes are not qualified to characterize the realistic base station distribution according to large-scale random verifications and multiple performance metrics. Details can be found in Chapter \ref{Chapter3}.

Since the two-dimensional point processes are not qualified to characterize the realistic base stations distribution, we turn to the one-dimensional spatial density distribution of base stations in cellular networks. Based on the large amount of real data, we conduct a general fitting process to find the most accurate statistical distribution of base stations density. After random area sampling in the under-investigated cellular networks, we choose several popular probability density function, including power-law, Weibull, log-normal and $\alpha$-Stable distribution along with the traditional Poisson distribution as candidates for the fitting procedure. According to the RMSE (root-mean-square-error) performance metric, we find that the $\alpha$-Stable distribution can best reflects the clustering nature of base stations distributions in cellular networks. Besides, along with the spatial distribution of BS, we try to connect the dots between the data traffic and the mobile user distribution. After examining the spatial distribution of these three dynamics, we find out that the spatial density of users, traffic and base stations are linearly growing with each other, revealing a trinity-like twisted phenomenon. Details can be found in Chapter \ref{Chapter4}.

After revealing the clustering nature of cellular networks in different dimensions, we want to make good use of these unconventional properties. For example, by combining the content preference of mobile users and the spatial clustering of base stations, we investigate the distributed probabilistic caching strategy in the clustered heterogeneous cellular networks, based on the cooperation between base stations in the same spatial group. Assuming that nearby base station pairs share a limited interchangeable bandwidth, the served-by-all BS collaboration helps to reduce the overall content delivery latency of the radio access network caching scenario.

Besides, considering the temporal burst of traffic demand and spatial clustering of base stations, there is a chance for the introduction of BS sleeping strategies into cellular networks to improve the overall energy efficiency. Actually, the basic principle is that the default traffic demand of base station which encounters a low-load situation can be served by the nearby ones with tolerable QoS and inexpensive cost because the base stations are spatial clusteringly distributed. Intuitively, in this cluster scenario, the overall performance will be better than that of the homogeneous case, as illustrated in our analysis.

On the other hand, by combining that online content preference of mobile users and the temporal burst of traffic demand, we can adopt the broadcasting technic in cellular networks, in order to diminish the number of wireless transmissions and meanwhile ensure the QoE requirement of users. In this case, the spatially clustering nature of mobile user is another necessity to make the broadcasting procedure practical and beneficial. Thus, three dimensional aggregation properties are all considered in this scenario, which can make a noticeable difference on the capacity improvement comparing to the traditional broadcasting technics in wireless networks. All these technics based on different combination of clustering dimension are introduce in Chapter \ref{Chapter5}.

Above all, given these clustering nature of mobile users, traffic demand and base stations on different dimensions (i.e. temporal, spatial and content), which are certificated by large amount of real data, a new paradigm of service procedure can be explored. That's to say, firstly analyzing the statistical characteristics of both mobile user and fixed infrastructure by collecting dynamic registration or static deployment records, then monitoring the overall traffic demand by collecting the online content requests distributively, finally go through a centralized processor with holistic information to make a globally optimized service solution. In fact, this kind of service paradigm needs high-performance data processing technics like data mining, traffic prediction methods, along with high-capacity central controller for fast and reliable instruction delivery and this task provides a appropriate scenario for big data technology and software defined networking paradigm. This new centralized architecture making use of distributed real measurement are introduced in Chapter \ref{Chapter6}.

\section{Contributions}

The contributions of this thesis can be concluded to several parts as following.

Firstly, we make a comprehensive review of the state-of-the-art real data measurement in cellular networks. Specifically, the related works cover those data analysis in mobile users, traffic aggregation and network infrastructure, on different dimensions including temporal, spatial and contents, also in different perspectives like resource allocation, QoS (Quality of Service) and energy efficiency. Most of these measurement results reveal kind of clustering nature in cellular networks.

Secondly, we conduct a large-scale identification of base stations in cellular networks specifically about the spatial distribution, separately on typically urban, rural sample region and on numbers of randomly selected regions. Choosing most of those popular point processes as fitting candidates, we try to find out the most appropriate spatial model for base station distribution. However, all these well understood two-dimensional point process models are rejected by the large amount of real data according to either classical statistical or coverage probability performance metrics. After all, instead of pursuing a two-dimensional point process model for spatial distribution, we turn to the one-dimensional BS density distribution characterization and find out that $\alpha$-Stable is the most accurate statistical distribution for BS and traffic demand density reflecting the explicit clustering nature of infrastructure in cellular networks.

Thirdly, to make full use of the clustering nature of user requested contents and the spatial distribution of base stations, we introduced collaborative caching strategy in two-tier heterogeneous cellular networks where small base stations within the same cluster can exchange contents with each other constrained by a given bandwidth. According to the simulation results, we find that the cooperation scheme can reduce the average content delivery latency of mobile users significantly. Besides, we investigate the impact of base station density on the caching performance, and provide a overall latency evaluation of the clusteringly distributed cellular networks.

Fourthly, considering the clustering nature on time and content dimension of mobile users traffic demand, we propose to adopt the broadcasting technics into cellular networks.

Fifthly, we put forward the intelligent controlling architecture which involves high-accuracy machine learning algorithms and high-reliability centralized processing units, to be embedded to the traditional cellular networks. In this intelligent paradigm, the static or dynamic records of mobile users, traffic demand and base stations can all be tracked for the real-time service strategy decision.

Conclusively, this thesis investigates the clustering nature of cellular networks in different dimensions and utilizing them to improve the service efficiency of several well known communication technologies. In addition to the theoretical and simulating analysis, we put forward a promising intelligent controlling architecture packed with machine learning algorithms and SDN (Software Defined Networking) technics to make full use of the clustering nature of cellular networks.

The presentation structure of this thesis is as following, Chapter \ref{Chapter2} gives a comprehensive introduction to the related works in cellular networks traffic measurements; Chapter \ref{Chapter3} presents a specific investigation on the spatial distribution of base stations based on two-dimensional point process modeling. Furthermore, the $\alpha$-Stable distribution is introduced to model the spatial density of BS, traffic demand, and temporal aggregation effect in Chapter \ref{Chapter4}. After that, the probabilistic caching strategy based on nearby base stations collaboration is analyzed in Chapter \ref{Chapter5}, along with which the broadcasting technics considering temporal clustering in cellular networks. Above all, after revealing the statistical results and analyzing the technology potentials, an overall intelligent controlling architecture is proposed in Chapter \ref{Chapter6}. After all, the conclusion is given in Chapter \ref{Chapter7}.





